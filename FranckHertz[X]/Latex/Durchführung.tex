\subsection{Energieverteilung der beschleunigten Elektronen}
Zur Messung der integralen Energieverteilung der beschleunigten Elektronen, wird der Auffängerstrom $I_A$ in Abhängigkeit von der Bremsspannung $U_A$ gemessen.\\
Die Beschleunigungsspannung $U_B$ wird bei +11 V konstant gehalten.\\ 
Es wird jeweils eine Messung bei T $\approx$ 20°C und zwischen T = 140° und 160°C durchgeführt. Das Ergebnis wird mit einem XY-Schreiber festgehalten (X-Eingang: Beschleunigungsspannung, Y-Eingang: Auffängerstrom).
\subsection{Aufnahme von Franck-Hertz-Kurven}
Für drei unterschiedliche Temperaturen zwischen 160° und 200° C werden mit Hilfe eines XY-Schreibers (X-Eingang: $U_B$, Y-Eingang: $I_A$) Franck-Hertz-Kurven aufgenommen.\\
Als Bremsspannung $U_A$ wird 1 V gewählt, während $U_B$ von 0V auf 60V steigt bzw. aufgetragen wird.
\subsection{Ionisierungsspannung von Hg-Atomen}
Zur Bestimmung der Ionisierungsspannung wird mit einem XY-Schreiber (X-Eingang: $U_B$, Y-Eingang: $I_A$) bei einer Temperatur zwischen 100°C und 110°C der Auffängerstrom $I_A$ in Abhängigkeit von der Beschleunigungsspannung $U_B$ aufgenommen. Die Bremsspannung $U_A$ liegt bei -30V.\\
