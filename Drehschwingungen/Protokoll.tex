\input{Latex/Preambel.tex}

\begin{document}

  \section{Einleitung}
  
  
  
  \section{Aufgaben}
  
  
  
    \subsection{Vorbereitungsaufgaben}
    
    
    
    \subsection{Augabenstellung}
    
    
    
  \section{Theorie}
  
  
  
  \section{Durchführung}
  
  
  
  \section{Auswertung}
   
  In Tabelle \ref{Tab1} sind die Messwerte für die Periodendauer der Schwingung ohne äußeres Magnetfeld, sowie das daraus berechnete zeitliche Mittel aufgeführt.

    
    
      \begin{table}[h]
        \begin{tabular}{|c||c|}
          \hline
          Periodendauer & Periodendauer \\ 
          $T[\si{\second}]$ & $T[\si{\second}]$\\
          \hline \hline
          \num{18,364(1)} & \num{18,330(1)}\\[1ex] \hline   
          \num{18,377(1)} & \num{18,333(1)}\\[1ex] \hline 
          \num{18,353(1)} & \num{18,343(1)}\\[1ex] \hline  
          \num{18,359(1)} & \num{18,320(1)}\\[1ex] \hline  
          \num{18,346(1)} & \num{18,342(1)}\\[1ex] \hline  
          \num{18,349(1)} & $\overline{T} = \num{18.3469(3)}$\\[1ex] \hline
    
    
        \end{tabular}
        \centering
        \caption{Gemessene Periodendauern ohne äußeres Magnetfeld}
        \label{Tab1}
      \end{table}
      
  


  
  \section{Diskussion}
  
  

\end{document}

