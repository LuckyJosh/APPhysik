Im Folgenden werden die in \cref{sec:Auswertung} erhaltenen Ergebnisse noch einmal 
abschließend diskutiert und dabei auf ihre Plausibilität hin überprüft.\\


Die Ergebnisse der Untersuchung der beiden Zerfälle des Rhodiums erweisen sich im Vergleich 
zu den Literaturwerten \cite{NuklidKarte_Rh} $T_{\sfrac{1}{2},l} = \SI{260.4}{\second}$ und
$T_{\sfrac{1}{2},k} = \SI{42.3}{\second}$ als plausibel. Die relative Abweichung der
längeren Halbwertszeit des $\ce{^{104}_{45}Rh}$ von den Literaturwerten ist mit $\Delta T_{\sfrac{1}{2},l} \approx 
\SI{7}{\percent}$ relativ gering wohin gegen der erhaltene Wert der Halbwertszeit für  $\ce{^{104i}_{45}Rh}$
mit $\Delta T_{\sfrac{1}{2},k} \approx \SI{29}{\percent}$ eine weitaus größere Abweichung aufweist.
Ähnlich verhält es sich mit den bestimmten statistischen Abweichungen der berechneten 
Halbwertszeiten. So entspricht die Abweichung des berechneten Wertes für $T_{\sfrac{1}{2},l}$ 
ca. \SI{60}{\percent} Ergebnisses und die Abweichung der Halbwertszeit $T_{\sfrac{1}{2},k}$ 
noch \SI{24}{\percent} von dieser.\\
Begründen lassen sich diese großen Abweichungen durch die statistischen Fehler der Messwerte,
die durch die Logarithmierung für kleine Messwerte größer ausfallen (vgl. \cref{std:ln}).
Da die Bestimmung der Halbwertszeit des langlebigeren Zerfalls erst ab $t^{*} = \SI{480}{\second}$
und somit für entsprechend kleine Werte für $\Ln{N(t)}$ durchgeführt wird, ist der statistische 
Fehler des erhaltenen Ergebnisses in gleicher Weise größer. Hinzu kommt noch ein ähnlicher Effekt  
der aus der Berechnung des Fehlers der Halbwertszeit \cref{std:Halbwertszeit} herrührt, wodurch der 
Fehler für kleine $\lambda$ ebenfalls stark vergrößert wird.
Dies gilt in gleichem Maße auch für die Abweichung des kurzlebigen Zerfalls, bei dessen Berechnung
jedoch die Ergebnisse des langlebigen Zerfalls verwendet werden und sich der Fehler somit auf diese
Ergebnisse überträgt. Die prozentual geringere Abweichung ist damit zu erklären, dass die Messwerte
für die Zeiten $t << t^{*}$ weniger große Abweichungen aufweisen.

Aus der \cref{fig:Auswertung_Rh_Ergebniss} ist jedoch zu erkennen, dass die Summer beider Zerfälle dem 
Verlauf der Messwerte in guter Näherung darstellt. Weiter ist gut zu erkennen, dass der summierte 
Zerfall bis zu dem gewählten Zeitpunkt $t^{*} = \SI{480}{\second}$ in einen linearen Verlauf 
übergeht und ab diesem dem Verlauf des langlebigeren Zerfalls entspricht.\\

Der ermittelte Wert der Halbwertszeit für $\ce{^{116}In}$ weicht in geringem Maße vom Literaturwert \cite{NuklidKarte_In} ab. Die relative Abweichung beträgt hierbei $\Delta_{rel}T_{1/2}=2.9\%$. Diese Abweichung lässt sich mit statistischer Fluktuation erklären. Allgemein ist dieses Ergebnis jedoch als recht präzise einzuordnen.  