
% 
%
% Allgemeine Theorie
%
%

\subsection{Elemente mit einem Zerfall}




\subsection{Elemente mit zwei unterschiedlichen Zerfällen}
	Neben den Elementen die durch die Aktivierung mit Neutronen nur einfach zerfallen, 
	wie das in diesem Versuch untersuchte \ce{^{116}_{49}In}, existieren auch Elemente
	die durch die Aktivierung in zwei unterschiedlichen weisen zerfallen.
	Als Beispiele sollen hier Silber und das in diesem Versuch untersuchte Rhodium 
	betrachtet werden um zwei unterschiedliche Begründungen für diesen Umstand zu 
	klären.
	
	Einen Grund für zwei unterschiedliche Zerfälle ist die Zusammensetzung
	der natürlichen Elemente aus verschiedenen Isotopen. So besteht das natürliche 
    Silber zu ungefähr gleichen Teilen aus den Isotopen \ce{^{107}_{47}Ag} und \ce{^{109}_{47}Ag},
    die beide durch die Neutronen aktiviert werden und sich für die dabei entstandenen 
    instabilen Kerne die beiden Zerfälle 
	\begin{subequations}
	 	\begin{empheq}{align}
	 	      &\ce{^{108}_{47}Ag -> ^{108}_{48}Cd + \beta - + \overline{\nu_{e}}}\\
	 	      &\ce{^{110}_{47}Ag -> ^{110}_{48}Cd + \beta - + \overline{\nu_{e}}}
	 	\end{empheq}
	\end{subequations}
    ergeben.
    
    Neben dieser gibt es noch eine weitere Begründung für das Ablaufen von zwei
    unterschiedlichen Zerfällen. Diese tritt bei dem in diesem Versuch untersuchtem
    Rhodium \ce{^{103}_{45}Rh} auf, welches nur aus einem natürlichen Isotop besteht.
    Bei der Aktivierung der Kerne dieses Elements entsteht in \SI{10}{\percent} der
    Fälle ein, zum sonst entstehenden \ce{^{104}_{45}Rh}, isomerer Kern \ce{^{104i}_{45}Rh}.
    Dieser unterscheidet sich nicht in der Anzahl sondern in der Konfiguration der 
    Nukleonen und damit in der Energie des Kerns. Es ergeben sich somit die 
    zwei möglichen Zerfälle der aktivierten Kerne
   	\begin{subequations}
   	 	\begin{empheq}{align}
   	 		  &\ce{^{104}_{45}Rh -> ^{104}_{46}Pd + \beta - + \overline{\nu_{e}}} \quad \text{und}\\
	 	      &\cee{^{104i}_{45}Rh -> ^{104}_{45}Rh + \gamma  \longrightarrow  ^{104}_{46}Pd + \beta - + \overline{\nu_{e}}}. 
   	 	\end{empheq}
   	\end{subequations}

	In beiden Fällen, sowohl bei Silber als auch bei Rhodium, können beide Zefälle gleichzeitig Untersucht werden.
	Dies ist zum einen möglich da, das verwendete Geiger-Müller-Zählrohr sowohl die  Betazerfälle der Silberisotope
    und des Rhodiumisotops \ce{^{104}_{45}Rh} als auch die Gammastrahlung des isomeren Kerns \ce{^{104i}_{45}Rh}
    nachweisen kann. Zum anderen besitzen die beiden Zerfälle jeweils eines Elements unterschiedliche Halbwertszeiten,
    so dass sie sich in der Auswertung des Versuchs  lassen und so getrennt von einander bestimmen lassen.\\   
    
    Zur Auswertung beider Zerfälle werden die logarithmierten Messwerte $\Ln{N(t)}$, wie in
     \cref{fig:Theorie_DoppelteAuswertung}, gegen die Zeit $t$ aufgetragen. 
    Da einer der beiden Zerfälle, wegen der unterschiedlichen Halbwertszeit, jeweils schneller abklingt
    als der andere, lässt sich ein Übergang des zunächst gekrümmten Verlaufs der Messwerte in ein linearen beobachten.
    An der Stelle dieses Übergangs wird der Zeitpunkt $t^{*}$ gewählt, ab dem die Messwerte wie bei einem 
    einfachen Zerfall linear ausgeglichen werden.
    
    Mit dem so bestimmten langlebigen Zerfall $N_{l}(t)$ lässt sich nun durch Subtraktion der Werte $N_{l}(t_{i})$
    für $t_{i} << t^{*}$ von den Messwerten $N(t)$ der Verlauf des kurzlebigen Zerfalls bestimmen, der analog 
 	zum langlebigen Zerfall linear ausgeglichen wird. Durch dieses Vorgehen erhält man nun den das Zerfallsgesetz
 	$N_{k}(t)$ der kurzlebigen Kerne.   
       
    \includeFigure[scale=0.7]{Grafiken/DoppelteAuswertung.png}{Graphische Darstellung des Vorgehens zur Auswertung von
    zwei gleichzeitig ablaufenden Zerfällen \cite{V702}}{\label{fig:Theorie_DoppelteAuswertung}}