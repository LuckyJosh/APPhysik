In diesem Abschnitt sind die zur Bestimmung der Mess- und Ergebnisfehler verwendeten
Fehlergleichungen zu finden. Die Gleichungen zur Bestimmung der Fehler von berechneten
Größen wurden dabei mit Hilfe der gaußschen Fehlerrechnung bestimmt. \\

Der Fehler der Messwerte für die Anzahl der Zerfälle $N$ ergibt sich  aus
\begin{errorEquation}
	\sigma_{N} = \sqrt{N}.
	\label{std:Abweichung Messwerte}
\end{errorEquation}


Den Fehler des Logarithmus  $\Ln{x}$ einer fehlerbehafteten Größe $x$ erhält man aus
\begin{errorEquation}
	 \sigma_{\ln} = \dfrac{\sigma_{x}}{x}. 
	 \label{std:ln}
\end{errorEquation}


Für den Wert der Exponentialfunktion $\E{x}$ berechnet sich der Fehler durch
\begin{errorEquation}
	 \sigma_{\exp} = \E{x} \cdot \sigma_{x}. 
	 \label{std:Exp}
\end{errorEquation}
 
 
 Der Fehler der Halbwertszeiten berechnet sich durch
 \begin{errorEquation}
 	 \sigma_{t_{\sfrac{1}{2}}} = \dfrac{\sigma_{\lambda} \cdot \Ln{2}}{\lambda^{2}}. 
 	 \label{std:Halbwertszeit}
 \end{errorEquation}