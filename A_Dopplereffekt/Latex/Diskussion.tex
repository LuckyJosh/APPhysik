Im Folgenden sollen die in \cref{sec:Auswertung} gewonnenen Messergebnisse noch einmal 
abschließend diskutiert und auf ihre Plausibilität hin untersucht werden.
Dabei werden auch der verwendete Versuchsaufbau und die Versuchsdurchführung mit diesen
in Bezug gesetzt.

Die in \cref{sec:Auswertung_FrequenzSchall} bestimmte Schallgeschwindigkeit zeigt eine Abweichung
von \SI{6(4)}{\percent} vom Literaturwert $c_{lit} = \SI{343}{\meter\per\second}$ \cite{Mende09} 
für trockene Luft bei \SI{20}{\celsius}. Diese Abweichung ist durch Ungenauigkeiten der 
Messungen für die Ruhefrequenz und -wellenlänge, welche sich vor allem in 
\cref{tab:Auswertung_Ruhefrequenz} zeigt. Ein Grund für diese Ungenauigkeit ist das 
kurze Messintervall von nur einer Sekunde, durch eine Messung über mehrere Sekunden 
würden diese Abweichungen durch das Mitteln über einen längeren Zeitraum verringert
wodurch eine genauere Messung möglich wäre.\\

Die Bestimmung der inversen Wellenlänge in den drei Teilversuchen in den Abschnitten
\ref{sec:Auswertung_Wellenlänge}, \ref{sec:Auswertung_Direkt} und \ref{sec:Auswertung_Schwebung}
Erfolgte in allen  drei Versuchen mittels einer anderen Messmethode und so ergeben sich 
für jede dieser Methoden unterschiedliche und unterschiedlich genaue Messwerte. 
Schon offensichtlich ist, dass der mit Hilfe der Schwebungsmethode bestimmte Wert
mit \SI{44}{\percent} zur Wellenlängenmessung und \SI{38}{\percent} zur direkten Messung
große Abweichungen aufweist. Während die Abweichung zwischen den andern beiden Messungen 
mit \SI{4}{\percent} nicht besonders groß ausfällt.
Der mit diesen Werten durchgeführte studentsche t-Test unterstreicht einerseits den 
beschriebenen ersten Eindruck, da die Prüfgrößen der beiden Vergleiche mit dem 
Ergebnis der Schwebungsmethode unverhältnismäßig viel höher sind als die jeweilige 
Referenzgröße. Somit zeigt auch der t-Test, dass mit hoher Wahrscheinlichkeit systematische
Fehler zwischen diesem und den anderen Werten vorliegen.
Andrerseits liefert der t-Test dieses Ergebnis auch für den Vergleich zwischen den anderen
beiden Werten, obwohl sich diese auf den ersten Blick nicht wesentlich unterschieden.\\
Gründe für die abweichenden Ergebnisse der drei Messungen sind in dem verwendeten 
Versuchsaufbau zu finden. Hier ist vor allem die Zugvorrichtung und das 
Fahrverhalten des Wagens zu nennen, da dieser vor allem bei den hohen Geschwindigkeiten
eher über die schiene schleift als zu rollen. Dies beeinflusst maßgeblich die Ergebnisse
der Schwebungsmessung, wodurch die in \cref{tab:Auswertung_Frequenz_Schwebung} zu sehenden 
Abweichungen auftreten.  
