In diesem Abschnitt werden die in \cref{sec:Auswertung} erhaltenen Ergebnisse noch einmal 
auf Plausibilität untersucht und mit dem Versuchsaufbau und der Versuchsdurchführung in Beziehung
gesetzt.\\


Die in \cref{sec:ohneNoise}  erhaltenen Ergebnisse scheinen Plausibel, da analog zu der theoretischen
Prognose eine Spannung $U_{out}$ gemessen wurde die proportional zur Signalspannung $U_{sig}$ ist.
Jedoch weisen die gemessen Werte gegen über der Theorie einen systematischen Fehler auf,
welcher sich in Form einer Phasen Verschiebung von $\Delta\phi = \SI{-90}{\degree}$ äußert. Durch diese
Verschiebung folgen die Messwerte einem sinusförmigen Verlauf, anstelle dem eines Kosinus.  
Dieser Fehler tritt auch bei der Messung mit Rauschen in \cref{sec:mitNoise} auf, wobei die Verschiebung
dabei $ \Delta\phi = \SI{90}{\degree}$ ist, sodass die Messwerte einem negativen Sinus folgen.
Eine Ursache für diesen Fehler zu finden ist aus den Messwerten, jedoch nicht möglich da es sich
wie zu vermuten ist um einen internen Fehler der verwendeten Apparatur handelt.

Ansonsten zeigen auch die in \cref{sec:Abstand} gemessenen Werte, dass der Lock-In-Verstärker 
auch schwache Signale, wie das Leuchten der LED auf größere Distanzen, auch unter erschwerten Bedingungen,
wie dem Umgebungslicht aufnehmen und wiedergeben kann.
 

