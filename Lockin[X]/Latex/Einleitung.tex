In dem Versuch \enquote{Der Lock-In-Verstärker} V303 soll die Funktionsweise eines Lock-In-Verstärkers verifiziert werden.
Dafür wird das Ausgangssignal nach Zuschalten einzelner Bauteile betrachtet und ausgewertet. Anschließend wird
mithilfe einer Photodetektorschaltung die Lichtintensität einer Lichtquelle in Abhängigkeit vom Abstand zwischen Quelle und Empfänger gemessen, um den Lock-In-Verstärker auf seine Rauschunterdrückung hin zu prüfen.
