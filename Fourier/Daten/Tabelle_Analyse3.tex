\begin{table}[!h]
	\centering
	\begin{tabular}{|c|c|c|c|}
		\hline
		Frequenzen & Gemessene Amplitude & Berechnete Amplitude & relative Abweichung \\
		$\nu\,[\si{\hertz}]$ & $b_{n}\,[\si{\volt}]$ & $b_{n,theo}\,[\si{\volt}]$ & $\envert{1 - \tfrac{b_{n}}{b_{n,theo}}}$ \\\hline\hline
		\num{100}  & \num{0.88}  & \num{1.27} & \num{0.31} \\
		\num{200}  & \num{0.45}  & \num{0.64} & \num{0.29} \\
		\num{300}  & \num{0.29}  & \num{0.42} & \num{0.32} \\
		\num{400}  & \num{0.24}  & \num{0.32} & \num{0.26} \\
		\num{500}  &\num{0.20}    & \num{0.26} & \num{0.22} \\
		\num{600}  & \num{0.15}  & \num{0.21} & \num{0.32} \\
		\num{700}  & \num{0.12}  & \num{0.18} & \num{0.34} \\
		\hline
	\end{tabular}
	\caption{Gemessene und Berechnete Amplituden der Oberschwingung\\ \hspace*{2.1cm}und deren relative Abweichung für die Sägezahnspannung \label{tab:Analyse3}}
\end{table}