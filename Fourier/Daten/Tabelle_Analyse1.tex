\begin{table}[!h]
	\centering
	\begin{tabular}{|c|c|c|c|}
		\hline
		Frequenzen & Gemessene Amplitude & Berechnete Amplitude & Relative Abweichung\\
		$\nu\,[\si{\hertz}]$ & $b_{n}\,[\si{\volt}]$ & $b_{n}\,[\si{\volt}]$ & $\envert{1 - \tfrac{b_{n}}{b_{n,theo}}}$\\\hline\hline
		\num{100}  & \num{1.80}  & \num{1.800}  & \num{0.00} \\
		\num{300}  & \num{0.60}  & \num{0.600}  & \num{0.00} \\
		\num{500}  & \num{0.34}  & \num{0.360}  & \num{0.06} \\
		\num{700}  & \num{0.25}  & \num{0.257}  & \num{0.03} \\
		\num{900}  & \num{0.19}  & \num{0.200}  & \num{0.05} \\
		\num{1100}  & \num{0.14}  & \num{0.164}  & \num{0.14} \\
		\hline
	\end{tabular}
	\caption{Gemessene und Berechnete Amplituden der Oberschwingung der Rechtspannung \label{tab:Analyse1}}
\end{table}