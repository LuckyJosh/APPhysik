Die allgemeine Gasgleichung ist eine fundamentale Gleichung der Thermodynamik 
mit:
\begin{empheq}{equation}
pV = RT
\end{empheq}
R ist die allgemeine Gaskonstante.
Die Clausius-Clapeyronische Gleichung 
\begin{empheq}{equation}
(V_D - V_F)dp = \frac{L}{T}\cdot dT
\end{empheq}
wird verwendet um Dampfdruckkurve eines Stoffes zu ermitteln,
weiterhin lässt sich sagen, dass die Gleichung nur schwer integrierbar ist, da alle Variablen kompliziert von T abhängen.
Für manche Temperaturbereiche ist die Integration jedoch vereinfacht möglich, damit folgt:
\begin{empheq}{equation}
ln(p) = -\frac{L}{R}\cdot\frac{1}{T} + const.
\end{empheq}
bzw.
\begin{empheq}{equation}
p = p_0 \cdot e^{-\frac{L}{R}\cdot\frac{1}{T}}
\end{empheq}
Aus der nicht integrierten Form kann man auch weiter umformen, sodass man 
\begin{empheq}{equation}
\left(p + \frac{a}{V^2}\right) = RT
\end{empheq}
erhält.\\
Für die Auswertung wird 
\begin{empheq}{equation}
L_i := L - L_a
\end{empheq}
definiert, wobei $L_a$ die Verdampfungswärme, 
die benötigt wird um das Volumen $V_F$ der Flüssigkeit auf das Volumen des Dampfes $V_D$ auszudehnen. 
