In diesem Unterabschnitt werden, die durch die Auswertung erhaltenen
Größen mit Literaturwerten verglichen, um eine Aussage über deren 
Richtigkeit machen zu können. Außerdem werden diese Vergleiche noch
einmal mit dem Versuchsaufbau und der Versuchsdurchführung in Bezug
gesetzt um eventuelle Fehler aufzuzeigen, die etwaig Abweichung der 
errechneten Größen von der Realität erklären können.\\

Um die in \autoref{sec:gemittelte Verdampfungswärme} 
bestimmte, gemittelte Verdampfungswärme $L$ für Drücke 
$p < \SI{1}{\bar}$ mit dem Literaturwert $L_{lit} = 
\SI{2256}{\joule\per\g}$ \cite{Mende09}
vergleichen zu können, muss diese mit der 
Molaren Masse eines Wassermoleküls 
$M(\ce{H2O}) = \SI{18}{\g\per\mole}$\footnote{Berechnet aus den molaren Massen der Komponenten \cite{Kuchling07}} multipliziert werden, um die Verdampfungswärme pro Gramm mit 
$L =\SI{2272(22)}{\joule\per\g} $ zu erhalten.
Offensichtlich weicht die aus den Messwerten bestimmte Verdampfungswärme nur wenig vom angegebenen Literaturwert ab, diese qualitative Beobachtung lässt sich durch bilden der 
relativen Abweichung 
$ \Delta_{r} L = \tfrac{\envert{L - L_{lit}}}{L_{lit}} \approx 
\num{0.007} = \SI{0.7}{\percent}$ quantifizieren.
Dies hohe Genauigkeit zeigt, dass die Messung in diesem Teilversuch quasi
ohne systematische oder grobe Fehler erfolgte. Es sei jedoch darauf hingewiesen, dass
diese hohe Maß an Übereinstimmung nur für die zwölf verwendeten und nicht für alle 32 
aufgenommen Messwertpaare gilt, da gerade die Messwerte mit den geringsten Abweichungen 
zur Regressionsgerade für die Auswertung verwendet wurden.\\
Aus der hohen Genauigkeit der gemittelten Verdampfungswärme lässt sich darauf schließen,
dass auch die in \autoref{sec:innereVerdampfungswärme} besitmmte innere Verdampfungswärme
$L_{i}$ eine entsprechend oder zumindest vergleichbar hohe Übereinstimmung mit der Realität zeigt.\\

Die in \autoref{sec:TAbhängig} bestimmte Temperaturabhängigkeit der Verdampfungswärme
stellt ein Polynom zweiten Grades, eine nach oben geöffnete Parabel dar. 
	Diese stellt für hohe Temperaturen $ T > \SI{375}{K} $ einen plausiblen Verlauf dar,
verliert jedoch an Plausibilität für geringere Temperaturen, da ein Ansteigen der
Verdampfungswärme sowohl für größere als auch für geringere Temperaturen nicht
realistisch ist.
Der Verlauf in Form eines Polynoms zweiten Grades lässt sich durch die Bestimmung
des Differentialquotienten $\tod{p}{T}$ durch Differentiation des Regressionsploynoms 
dritten Grades aus \autoref{fig:pT2} erklären.\\


Zusammenfassend lässt sich sagen, dass die in diesem Versuch vorgenommenen Messungen 
eine, im Vergleich zur Literatur, geringe Abweichung aufwiesen, wodurch auf eine allgemein niedrige Fehleranfälligkeit des Versuchs zuschließen ist. 