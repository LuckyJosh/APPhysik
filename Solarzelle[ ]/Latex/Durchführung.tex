
Vor Beginn des eigentlichen Versuchs werden die Maße der Solarzelle und die Offsets der 
des Abstandes von Lampe und Solarzelle aufgenommen. Dabei ist der erste Offset $d_{\text{off},1}$
der Abstand von der Solarzelle zur Steckplatte und der zweite $d_{\text{off},2}$ der Abstand der 
Lampe vom Nullpunkt der Skala.
Für die Durchführung wir die Solarzelle dem Schaltplan in \cref{fig:Durchfuehrung_Schaltplan}
entsprechend mit den Multimetern und der Widerstandsdekade verbunden.

\includeFigure[scale=0.5]{Grafiken/Schaltplan}{Plan der in diesem Versuch verwendeten Schaltung %
\cite{NHV1}}{\label{fig:Durchfuehrung_Schaltplan}} 

Zunächst werden der Kurzschlussstrom und die Leerlaufspannung für unterschiedliche Abstände
von Solarzelle und Halogenlampe bestimmt. Für die Messung des Kurzschlussstromes  
wird dabei nur ein Ampermeter an die Solarzelle angeschlossen und der Strom $I_{K}$ in Abhängigkeit
des Abstandes $d$ notiert.
Entsprechend wir bei der Messung des Leerlaufspannung nur ein Voltmeter an die Solarzelle
angeschlossen, dessen angezeigte Werte $U_{L}$ ebenfalls in Abhängigkeit des Abstandes $d$ notiert werden.\\

Die Messung der $I\text{-}U$-Kennlinie der Solarzelle erfolgt für vier unterschiedliche Abstände in denen 
der Kurzschlussstrom jeweils einen der Werte \SI{30}{\milli\ampere}, \SI{50}{\milli\ampere}, \SI{75}{\milli\ampere} und 
\SI{100}{\milli\ampere} annimmt.
Bei einer Messung werden jeweils Strom $I_{SZ}$ und Spannung $U$ in Abhängigkeit des an der Widerstandsdekade 
eingestellten Widerstands $R$ zwischen \SI{1}{\ohm} und \SI{250}{\ohm} aufgenommen.