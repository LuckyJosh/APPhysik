Im folgenden Abschnitt sind die während des Versuchs aufgenommenen Messwerte
und die aus diesen berechneten Ergebnisse tabellarisch und grafisch dargestellt.
An entsprechender Stelle sind Anmerkungen und Erklärungen zu den Berechnungen und
Ergebnissen gegeben. 


\subsection{Bestimmung der Plateau-Steigung des Zählrohrs}

	Die für die Zählrohr-Charakteristik aufgenommenen Messwerte für
	Spannung $U$ und Anzahl der registrierten Impulse $P$ in dem Zeitintervall
	$\Delta t = \SI{100}{\second}$ sind in \cref{tab:} zu finden.
	
%	\begin{table}[!h]
	\centering
	\begin{tabular}{|c|c||c|c|}
		\hline
		Spannung & Anzahl der Pulse & Spannung & Anzahl der Pulse\\
		$U$ [\si{\volt}] & $Z$ & $U$ [\si{\volt}] & $Z$ \\
\hline\hline
		\num{350(1)} & \num{5036(70)} & \num{530(1)} & \num{5354(73)}\\
		\num{370(1)} & \num{4995(70)} & \num{550(1)} & \num{5002(70)}\\
		\num{390(1)} & \num{4966(70)} & \num{570(1)} & \num{5162(71)}\\
		\num{410(1)} & \num{5040(70)} & \num{590(1)} & \num{5185(72)}\\
		\num{430(1)} & \num{5252(71)} & \num{610(1)} & \num{5184(72)}\\
		\num{450(1)} & \num{5137(71)} & \num{630(1)} & \num{5292(72)}\\
		\num{470(1)} & \num{5049(71)} & \num{650(1)} & \num{5033(70)}\\
		\num{490(1)} & \num{5197(72)} & \num{670(1)} & \num{5297(72)}\\
		\num{510(1)} & \num{5076(71)} & \num{690(1)} & \num{5153(71)}\\
		\hline
	\end{tabular}
	\caption{Messwerte für die Charakteristik des Zählrohrs \label{tab:Auswertung_Charakteristik}}
\end{table}

	
	Diese Messwerte sind in \cref{fig:} grafisch dargestellt. Die farbigen
	Messwerte wurden für eine lineare Regression mit dem Ansatz
	
	\begin{empheq}{equation}
		P(U) = A \cdot U + P_0
	\end{empheq}  
	verwendet. Die Durchführung der Regression unter Verwendung der \emph{Python}-Bibliothek 
	\emph{SciPy} ergab die Parameter
	
	\addtocounter{equation}{-1}
	\begin{subequations}
		\begin{empheq}{align}
			A &= \SI{0.98(7)}{\per\volt}\quad \text{und}\\	
			P_0 &= \SI{4631(34)}{}.	
		\end{empheq}
	\end{subequations}

	Somit ergibt sich die Plateau-Steigung des Zählrohrs zu
	\begin{empheq}{equation}
		A = (\num{98(7)}) \dfrac{\si{\percent}}{100\si{\volt}}.
	\end{empheq}	
	
	
\subsection{Bestimmung des zeitlicher Abstand zwischen Primär- und Nachentladungimpuls}
	
	Auf dem mit dem Zählrohr verbundenen Oszilloskop, lässt sich die Primärentladung als 
	annähernd stehendes Bild einer Kurve mit der in Abb.3 \cite{V703} dargestellten Form 
	ausmachen. Die Nachentladungen sind ebenfalls Kurven dieser Form, mit dem Unterschiede
	das diese nur für eine sehr kurze Zeit auf dem Bildschirm des Oszilloskops angezeigt werden.
	
	Der mittlere zeitliche Abstand zwischen beiden Entladungen konnte, aufgrund der kurzen Zeitspanne 
	in der beide Entladungen zu sehen sind, nur auf den Wert
	\begin{empheq}{equation}
		\Delta t_{P,N} \approx \SI{150}{\micro\second}
	\end{empheq} 
	geschätzt werden.
	 

\subsection{Bestimmung der Totzeit mit Hilfe eines Oszilloskops}

	Die Totzeit des Zählrohrs entspricht nach Abb.3 \cite{V703} in etwa der 
	Breite des Primärladungsimpulses, diese wurde zu
	\begin{empheq}{equation}
		T \approx \SI{150}{\micro\second}
	\end{empheq} 
	bestimmt.
	

\subsection{Bestimmung der Totzeit nach der Zwei-Quellen-Methode}

	In sind die aufgenommenen Werte für die Impulsraten der ersten Quelle
	$N_{1}$, der zweiten Quelle $N_{2}$ und beider Quellen $N_{1+2}$ gelistet.
	
%	\begin{table}[!h]
	\centering
	\begin{tabular}{|c|c|c|}
		\hline
		Impulsrate 1 & Impulsrate 2 & Impulsrate 1+2\\
		$N_1$ [\si{\per\second}] & $N_2$ [\si{\per\second}] & $N_{1+2}$ [\si{\per\second}]\\
\hline\hline
		\num{10.3(3)} & \num{73.9(8)} & \num{82.7(8)}\\
		\hline
	\end{tabular}
	\caption{Aufgenommene Impulsraten der Einzelquellen und beider Quellen zusammen \label{tab:Auswertung_Totzeit}}
\end{table}


	Nach \cref{eq:Theorie_Totzeit} ergibt sich die Totzeit des Zählrohrs 
	rechnerisch zu 
	\begin{empheq}{equation}
		T = \SI{900(800)}{\micro\second}.
	\end{empheq}	
	
	
