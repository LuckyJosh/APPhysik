Im folgenden Abschnitt sind die während des Versuchs aufgenommenen Messwerte
und die aus diesen berechneten Ergebnisse tabellarisch und grafisch dargestellt.
An entsprechender Stelle sind Anmerkungen und Erklärungen zu den Berechnungen und
Ergebnissen gegeben. Die für die Fehlerrechnung verwendeten Gleichungen sind in 
\cref{sec:Fehlerrechnung} angegeben. Die Fehler für Messwerte werden im Allgemeinen durch
die kleinste Skaleneinteilung des verwendeten Messgeräts abgeschätzt.

\subsection{Fehlerrechnung}\label{sec:Fehlerrechnung}
			Im folgenden Abschnitt sind die, für die Fehlerrechnung genutzten,
Gleichungen aufgelistet, die mit Hilfe der gaußschen Fehlerfortpflanzung
bestimmt wurden.\\

Der Fehler einer poissonverteilten Größe, wie die Anzahl der radioaktiven Zerfälle
ergibt sich durch 
\begin{errorEquation}
\label{std:Zerfall}
\sigma_{Z} = \sqrt{Z}.
\end{errorEquation}   

Die Fehler der Impulsraten erhält man aus den Fehlern der Impulsanzahl
durch
\begin{errorEquation}
\label{std:Impulsrate}
\sigma_{N} = \dfrac{\sigma_{P}}{\Delta t}.
\end{errorEquation}   


Der Fehler der rechnerisch bestimmten Totzeit ergibt sich 
nach
\begin{errorEquationAlign}
\notag
\label{std:Totzeit}
\sigma_{T}^{2}=\frac{N_{1}^{2} N_{2}^{2}}{4} \sigma_{N_{{1+2}}}^{2} &+ \sigma_{N_{1}}^{2} \left(\frac{N_{1} N_{2}}{2} + N_{2} \left(\frac{N_{1}}{2} + \frac{N_{2}}{2} - \frac{N_{{1+2}}}{2}\right)\right)^{2}  \\
 &+ \sigma_{N_{2}}^{2} \left(\frac{N_{1} N_{2}}{2} + N_{1} \left(\frac{N_{1}}{2} + \frac{N_{2}}{2} - \frac{N_{{1+2}}}{2}\right)\right)^{2}.
\end{errorEquationAlign}   


Den Fehler der freigesetzten Ladungsmenge pro einfallendem Teilchen erhält man durch
\begin{errorEquation}
\label{std:Ladungsmenge}
\sigma_{\Delta Q} = \sqrt{\del{\dfrac{\sigma_{\overline{I}}}{N}}^{2}  + \del{\dfrac{\overline{I}\sigma_{N}}{N^{2}}}^{2}}.
\end{errorEquation}

\subsection{Bestimmung der Plateau-Steigung des Zählrohrs}

	Die für die Zählrohr-Charakteristik aufgenommenen Messwerte für
	Spannung $U$ und Anzahl der registrierten Impulse $Z$ in dem Zeitintervall
	$\Delta t = \SI{100}{\second}$ sind in \cref{tab:Auswertung_Charakteristik} zu finden.
	
	\begin{table}[!h]
	\centering
	\begin{tabular}{|c|c|c||c|c|c|}
		\hline
		Spannung & Anzahl der Pulse & Spannung & Anzahl der Pulse\\
		$U$ [\si{\volt}] & $P$ &  $U$ [\si{\volt}] & $P$ \\
\hline\hline
		\num{350(1)} & \num{5036(70)} & \num{530(1)} & \num{5354(73)}\\
		\num{370(1)} & \num{4995(70)}  & \num{550(1)} & \num{5002(70)}\\
		\num{390(1)} & \num{4966(70)} &   \num{570(1)} & \num{5162(71)}\\
		\num{410(1)} & \num{5040(70)} &   \num{590(1)} & \num{5185(72)}\\
		\num{430(1)} & \num{5252(72)}  & \num{610(1)} & \num{5184(72)} \\
		\num{450(1)} & \num{5137(71)}  & \num{630(1)} & \num{5292(72)} \\
		\num{470(1)} & \num{5049(71)}  & \num{650(1)} & \num{5033(70)} \\
		\num{490(1)} & \num{5197(72)}  & \num{670(1)} & \num{5297(72)} \\
		\num{510(1)} & \num{5076(71)}  & \num{690(1)} & \num{5153(71)} \\
		\hline
	\end{tabular}
	\caption{Messwerte für die Charakteristik des Zählrohrs \label{tab:Auswertung_Charakteristik}}
\end{table}

	
	Diese Messwerte sind in \cref{fig:Auswertung_Charakteristik} grafisch dargestellt. Die 
	Messwerte wurden für eine lineare Regression mit dem Ansatz
	
	\begin{empheq}{equation}
		Z(U) = A \cdot U + Z_0
	\end{empheq}  
	verwendet. Die Durchführung der Regression unter Verwendung der \emph{Python}-Bibliothek 
	\emph{SciPy} ergab die Parameter
	
	\addtocounter{equation}{-1}
	\begin{subequations}
		\begin{empheq}{align}
			A &= \SI{0.5(2)}{\per\volt}\quad \text{und}\\	
			Z_0 &= \SI{4860(120)}{}.	
		\end{empheq}
	\end{subequations}

	Die prozentuale Änderung der Impulsanzahl pro \SI{100}{\volt}, welche die übliche Angabe für 
	die Plateau-Steigung des Zählrohrs ist, ergibt sich somit zu
	\begin{empheq}{equation}
		\dfrac{Z(\SI{100}{\volt}) - Z_0}{Z_0} \dfrac{1}{\SI{100}{\volt}} = (\num{1.1(5)}) \dfrac{\si{\percent}}{100\si{\volt}}.
	\end{empheq}	
	
	\includeFigure[scale=0.7]{Grafiken/Charakteristik.pdf}{Grafische Darstellung der
	 aufgenommenen Impulse in Abhängigkeit der Spannung mit linearer Regression des Plateaus}{
	 \label{fig:Auswertung_Charakteristik}}
	
	
\subsection{Bestimmung des zeitlicher Abstand zwischen Primär- und Nachentladungimpuls}
	
	Auf dem mit dem Zählrohr verbundenen Oszilloskop, lässt sich die Primärentladung als 
	annähernd stehendes Bild einer Kurve mit der in \cref{Totzeit} dargestellten Form 
	ausmachen. Die Nachentladungen sind ebenfalls Kurven dieser Form, mit dem Unterschiede
	das diese nur für eine sehr kurze Zeit auf dem Bildschirm des Oszilloskops angezeigt werden.
	
	Der mittlere zeitliche Abstand zwischen beiden Entladungen konnte, aufgrund der kurzen Zeitspanne 
	in der beide Entladungen zu sehen sind, nur auf den Wert
	\begin{empheq}{equation}
		\label{val:Primaer_Nachentladung_Oszilloskop}
		\Delta t_{P,N} \approx \SI{150}{\micro\second}
	\end{empheq} 
	geschätzt werden.
	 

\subsection{Bestimmung der Totzeit mit Hilfe eines Oszilloskops}

	Die Totzeit des Zählrohrs entspricht nach \cref{Totzeit} in etwa der 
	Breite des Primärladungsimpulses, diese wurde zu
	\begin{empheq}{equation}
		\label{val:Totzeit_Oszilloskop}
		T \approx \SI{150}{\micro\second}
	\end{empheq} 
	bestimmt.
	

\subsection{Bestimmung der Totzeit nach der Zwei-Quellen-Methode}

	In \cref{tab:Auswertung_Totzeit} sind die aufgenommenen Werte für die Impulsraten der ersten Quelle
	$N_{1}$, der zweiten Quelle $N_{2}$ und beider Quellen $N_{1+2}$ gelistet.
	
	\begin{table}[!h]
	\centering
	\begin{tabular}{|c|c|c|}
		\hline
		Impulsrate 1 & Impulsrate 2 & Impulsrate 1+2\\
		$N_1$ [\si{\per\second}] & $N_2$ [\si{\per\second}] & $N_{1+2}$ [\si{\per\second}]\\
\hline\hline
		\num{10.3(3)} & \num{73.9(8)} & \num{82.7(8)}\\
		\hline
	\end{tabular}
	\caption{Aufgenommene Impulsraten der Einzelquellen und beider Quellen zusammen \label{tab:Auswertung_Totzeit}}
\end{table}


	Nach \cref{eq:Theorie_Totzeit} ergibt sich die Totzeit des Zählrohrs 
	rechnerisch zu 
	\begin{empheq}{equation}
		T = \SI{900(800)}{\micro\second}.
		\label{val:Totzeit_Berechnet}	
	\end{empheq}	
	Der Fehler wurde dabei mittels \cref{std:Totzeit} bestimmt.
	
\subsection{Bestimmung der Ladungsmenge pro eingefallenem Teilchen}	
	
	Die Messwerte für die Stromstärke $\overline{I}$ bei der jeweiligen Spannung
	$U$ und die Werte für die Impulsrate $N = \sfrac{Z}{\Delta t}$ sind zusammen mit
	den, durch Umstellen von \cref{eq:Theorie_Ladungsmenge}, berechneten Ladungsmengen
	$\Delta Q$ in \cref{tab:Auswertung_Ladungsmenge} eingetragen.
	
	
	\begin{table}[!h]
	\centering
%	\begin{adjustbox}{width=\textwidth, center}
	\begin{tabular}{|c|c|c|c|}
		\hline
		Spannung & Stromstärke & Impulserate \cref{std:Impulsrate}& Ladungsmenge \cref{std:Ladungsmenge} \\
		$U$ [\si{\volt}] & $\overline{I}$ [\si{\micro\ampere}] & $N$ [\si{\per\second}] & $\Delta Q$ [\si{\giga e}]\\
\hline\hline
		\num{350(1)} & \num{0.1(1)} & \num{50.4(7)} & \num{12(12)} \\
		\num{370(1)} & \num{0.1(1)} & \num{50.0(7)} & \num{12(12)} \\
		\num{390(1)} & \num{0.1(1)} & \num{49.7(7)} & \num{12(12)} \\
		\num{410(1)} & \num{0.2(1)} & \num{50.4(7)} & \num{24(12)} \\
		\num{430(1)} & \num{0.2(1)} & \num{52.5(7)} & \num{23(12)}  \\
		\num{450(1)} & \num{0.2(1)} & \num{51.4(7)} & \num{23(12)} \\
		\num{470(1)} & \num{0.2(1)} & \num{50.5(7)} & \num{24(12)}  \\
		\num{490(1)} & \num{0.2(1)} & \num{52.0(7)} & \num{24(12)}  \\
		\num{510(1)} & \num{0.2(1)} & \num{50.8(7)} & \num{24(12)} \\
		\num{530(1)} & \num{0.3(1)} & \num{53.5(7)} & \num{34(12)} \\
		\num{550(1)} & \num{0.3(1)} & \num{50.0(7)} & \num{37(12)}\\
		\num{570(1)} & \num{0.3(1)} & \num{51.6(7)} & \num{36(12)} \\
		\num{590(1)} & \num{0.3(1)} & \num{51.9(7)} & \num{36(12)}\\
		\num{610(1)} & \num{0.4(1)} & \num{51.8(7)} & \num{48(12)}  \\
		\num{630(1)} & \num{0.4(1)} & \num{52.9(7)} & \num{47(12)}  \\
		\num{650(1)} & \num{0.4(1)} & \num{50.3(7)} & \num{49(12)} \\
		\num{670(1)} & \num{0.4(1)} & \num{53.0(7)} & \num{47(12)}\\
		\num{690(1)} & \num{0.4(1)} & \num{51.5(7)} & \num{48(12)}\\
		
		 
		\hline
	\end{tabular}
%	\end{adjustbox}
	\caption{Aufgenommene Stromstärken und Impulsraten zu der jeweilig anliegenden Spannung \label{tab:Auswertung_Ladungsmenge}}
\end{table}


















