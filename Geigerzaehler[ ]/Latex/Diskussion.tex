Im Folgenden werden die in \cref{sec:Auswertung} erhaltenen Ergebnisse noch einmal 
abschließend diskutiert und dabei auf ihre Plausibilität hin überprüft.
Dabei wird auch auf Versuchsdurchführung und -aufbau Bezug genommen.\\


Die vom Oszilloskop abgelesenen Werte für Totzeit \cref{val:Totzeit_Oszilloskop} und den zeitlichen Abstand
zwischen Primär- und Nachentladung \cref{val:Primaer_Nachentladung_Oszilloskop} sind nur als
Schätzungen zu verstehen, da eine genaue Messung aufgrund der kurzen Anzeigezeit nicht möglich war.
Dies zeigt sich auch in dem großen Unterschied zwischen der abgelesenen und der berechneten 
Totzeit \cref{val:Totzeit_Berechnet}. 

Ein weiteres Problem stellt der Fehler der berechneten Totzeit dar.
Der Grund für diese große Abweichung von rund \SI{89}{\percent}
ist die geringe Aktivität eines der verwendeten Beta-Strahler, wodurch 
der relative Fehler groß wird und bei der Berechnung der Totzeit zu einem entsprechend
hohen Fehler führt.

Die erhaltenen Werte sind somit im Allgemeinen plausibel, jedoch in Bezug auf die Genauigkeit nicht belastbar.  
  