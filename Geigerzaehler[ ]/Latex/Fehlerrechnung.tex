Im folgenden Abschnitt sind die, für die Fehlerrechnung genutzten,
Gleichungen aufgelistet, die mit Hilfe der gaußschen Fehlerfortpflanzung
bestimmt wurden.\\

Der Fehler einer poissonverteilten Größe, wie die Anzahl der radioaktiven Zerfälle
ergibt sich durch 
\begin{errorEquation}
\label{std:Zerfall}
\sigma_{Z} = \sqrt{Z}.
\end{errorEquation}   

Die Fehler der Impulsraten erhält man aus den Fehlern der Impulsanzahl
durch
\begin{errorEquation}
\label{std:Impulsrate}
\sigma_{N} = \dfrac{\sigma_{P}}{\Delta t}.
\end{errorEquation}   


Der Fehler der rechnerisch bestimmten Totzeit ergibt sich 
nach
\begin{errorEquationAlign}
\notag
\label{std:Totzeit}
\sigma_{T}^{2}=\frac{N_{1}^{2} N_{2}^{2}}{4} \sigma_{N_{{1+2}}}^{2} &+ \sigma_{N_{1}}^{2} \left(\frac{N_{1} N_{2}}{2} + N_{2} \left(\frac{N_{1}}{2} + \frac{N_{2}}{2} - \frac{N_{{1+2}}}{2}\right)\right)^{2}  \\
 &+ \sigma_{N_{2}}^{2} \left(\frac{N_{1} N_{2}}{2} + N_{1} \left(\frac{N_{1}}{2} + \frac{N_{2}}{2} - \frac{N_{{1+2}}}{2}\right)\right)^{2}.
\end{errorEquationAlign}   


Den Fehler der freigesetzten Ladungsmenge pro einfallendem Teilchen erhält man durch
\begin{errorEquation}
\label{std:Ladungsmenge}
\sigma_{\Delta Q} = \sqrt{\del{\dfrac{\sigma_{\overline{I}}}{N}}^{2}  + \del{\dfrac{\overline{I}\sigma_{N}}{N^{2}}}^{2}}.
\end{errorEquation}