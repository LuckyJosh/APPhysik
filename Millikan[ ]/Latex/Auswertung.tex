In folgendem Abschnitt sind er erhaltenen Messwerte und daraus berechneten
Ergebnisse tabellarisch und grafisch dargestellt.
An entsprechender Stelle sind Anmerkungen und Erklärungen zu den Berechnungen 
und Ergebnissen gegeben.\\

Die Messwerte\footnote{Bei der Durchführung des Versuch konnten keine brauchbaren
Messwerte aufgenommen werden, daher wurden uns die Messwerte einer anderen Gruppe zur 
Verfügung gestellt.} zu diesem Versuch, die Spannung $U$, die Steig- und Fallzeiten $t_{\text{auf}}$ und 
$t_{\text{ab}}$, sowie der Thermistor $R$ sind in \cref{tab:Auswertung_Messwerte} dargestellt. 
Wenn zwei Steig- oder Fallzeiten gemessen werden konnten wurde der Mittelwert aus diesen berechnet,
sonst entspricht der Mittelwert dem einzelnen Messwert.
 
 
 \begin{table}[!h]
	\centering
	\begin{adjustbox}{width=\textwidth,center}
		\begin{tabular}{|c|c|c|c|c|c|c|c|}
		\hline
		Spannung & Steigzeit 1 & Steigzeit 2 & Fallzeit 1 & Fallzeit 2 & Steigzeit Mittel & Fallzeit Mittel & Thermistor\\
		$U$ [\si{\volt}] & $t_{1,\text{auf}}$ [\si{\second}] & $t_{2,\text{auf}}$ [\si{\second}] & $t_{1,\text{ab}}$ [\si{\second}] & $t_{2,\text{ab}}$ [\si{\second}] & $\overline{t_{\text{auf}}}$ [\si{\second}] & $\overline{t_{\text{ab}}}$ [\si{\second}] & $R$ [\si{\mega\ohm}]\\
\hline\hline
		\num{298} & \num{4.635} & \num{0.000} & \num{4.412} & \num{0.000} & \num{4.635} & \num{4.412} & \num{1.870}\\
		\num{298} & \num{3.384} & \num{0.000} & \num{3.569} & \num{3.656} & \num{3.384} & \num{3.612} & \num{1.870}\\
		\num{298} & \num{6.753} & \num{6.712} & \num{7.759} & \num{8.199} & \num{6.732} & \num{7.979} & \num{1.840}\\
		\num{298} & \num{3.981} & \num{0.000} & \num{3.447} & \num{0.000} & \num{3.981} & \num{3.447} & \num{1.830}\\
		\num{297} & \num{3.611} & \num{3.366} & \num{3.296} & \num{3.472} & \num{3.489} & \num{3.384} & \num{1.710}\\
		\num{297} & \num{4.999} & \num{4.767} & \num{3.671} & \num{4.025} & \num{4.883} & \num{3.848} & \num{1.720}\\
		\num{297} & \num{3.609} & \num{0.000} & \num{3.480} & \num{3.266} & \num{3.609} & \num{3.373} & \num{1.710}\\
		\num{297} & \num{4.436} & \num{4.798} & \num{4.403} & \num{4.552} & \num{4.617} & \num{4.477} & \num{1.710}\\
		\num{297} & \num{5.113} & \num{4.618} & \num{0.000} & \num{5.580} & \num{4.866} & \num{5.580} & \num{1.730}\\
		\num{297} & \num{7.816} & \num{6.635} & \num{7.000} & \num{7.773} & \num{7.226} & \num{7.386} & \num{1.720}\\
		\num{297} & \num{5.768} & \num{5.935} & \num{7.482} & \num{7.830} & \num{5.851} & \num{7.656} & \num{1.720}\\
		\num{297} & \num{3.138} & \num{0.000} & \num{3.270} & \num{3.340} & \num{3.138} & \num{3.305} & \num{1.810}\\
		\num{297} & \num{3.384} & \num{3.287} & \num{2.870} & \num{3.576} & \num{3.335} & \num{3.223} & \num{1.820}\\
		\num{297} & \num{3.812} & \num{3.380} & \num{3.206} & \num{3.814} & \num{3.596} & \num{3.510} & \num{1.810}\\
		\num{201} & \num{4.521} & \num{0.000} & \num{4.153} & \num{0.000} & \num{4.521} & \num{4.153} & \num{1.740}\\
		\num{201} & \num{2.824} & \num{0.000} & \num{2.683} & \num{2.584} & \num{2.824} & \num{2.633} & \num{1.740}\\
		\num{201} & \num{4.183} & \num{4.462} & \num{4.636} & \num{5.260} & \num{4.322} & \num{4.948} & \num{1.770}\\
		\num{201} & \num{12.915} & \num{0.000} & \num{14.231} & \num{0.000} & \num{12.915} & \num{14.231} & \num{1.770}\\
		\num{201} & \num{5.042} & \num{0.000} & \num{5.580} & \num{5.249} & \num{5.042} & \num{5.415} & \num{1.740}\\
		\num{201} & \num{4.439} & \num{4.420} & \num{5.886} & \num{4.915} & \num{4.429} & \num{5.401} & \num{1.800}\\
		\num{201} & \num{6.134} & \num{6.155} & \num{5.801} & \num{5.596} & \num{6.145} & \num{5.699} & \num{1.790}\\
		\num{201} & \num{10.338} & \num{0.000} & \num{8.710} & \num{0.000} & \num{10.338} & \num{8.710} & \num{1.790}\\
		\num{201} & \num{7.092} & \num{0.000} & \num{5.312} & \num{5.637} & \num{7.092} & \num{5.474} & \num{1.780}\\
		\num{201} & \num{17.314} & \num{0.000} & \num{14.080} & \num{0.000} & \num{17.314} & \num{14.080} & \num{1.790}\\
		\num{201} & \num{12.156} & \num{0.000} & \num{16.894} & \num{0.000} & \num{12.156} & \num{16.894} & \num{1.780}\\
		\hline
		\end{tabular}
	\end{adjustbox}
	\caption{Die aufgenommenen Messwerte für die Steig- und Fallzeiten der Öltröpfchen, deren Mittelwerte und der Wert des Thermowiderstands während der jeweiligen Messung \label{tab:Auswertung_Messwerte}}
\end{table}




Über die Verwendete Messstrecke $s = \SI{0.5}{\mm}$ wurde aus den Zeiten $t_{\text{auf}}$ und $t_{\text{ab}}$ die
korrespondierenden Geschwindigkeiten $v_{\text{auf}}$ und $v_{\text{ab}}$ berechnet. Die Temperaturen der Luft 
$T$ konnten unter Verwendung der Thermistormesswerte aus der Tabelle 1 in \cite{V503} abgelesen und
wiederum zur Bestimmung der Luftviskosität $\eta_{L}$ aus Abbildung 3 in \cite{V503} verwendet werden.
Diese Werte befinden sich zusammen mit der im weiteren Verlauf gebrauchten Differenz der Geschwindigkeiten 
$v_{\text{ab}} - v_{\text{auf}}$ in \cref{tab:Auswertung_Ergebnisse}.  
 
	\begin{table}[!h]
	\centering
	\begin{adjustbox}{width=\textwidth, center}
	\begin{tabular}{|c|c|c|c|c|}
		\hline
		Steiggeschwindigkeit & Fallgeschwindigkeit & Differenzgeschwindigkeit & Lufttemperatur & Luftviskosität\\
		$v_{\text{auf}}$ [\si{\milli\meter\per\second}] & $v_{\text{ab}}$ [\si{\milli\meter\per\second}] & $v_{\text{ab}} - v_{\text{auf}}$ [\si{\milli\meter\per\second}] & $T$ [\si{\celsius}] & $\eta_{L}$ [\si{\micro\newton\second\per\square\meter}]\\
\hline\hline
		\num{0.108} & \num{0.113} & \num{0.005} & \num{28} & \num{18.620}\\
		\num{0.148} & \num{0.138} & \num{-0.009} & \num{28} & \num{18.620}\\
		\num{0.074} & \num{0.063} & \num{-0.012} & \num{28} & \num{18.620}\\
		\num{0.126} & \num{0.145} & \num{0.019} & \num{29} & \num{18.670}\\
		\num{0.143} & \num{0.148} & \num{0.004} & \num{32} & \num{18.820}\\
		\num{0.102} & \num{0.130} & \num{0.028} & \num{31} & \num{18.770}\\
		\num{0.139} & \num{0.148} & \num{0.010} & \num{32} & \num{18.820}\\
		\num{0.108} & \num{0.112} & \num{0.003} & \num{32} & \num{18.820}\\
		\num{0.103} & \num{0.090} & \num{-0.013} & \num{31} & \num{18.770}\\
		\num{0.069} & \num{0.068} & \num{-0.002} & \num{31} & \num{18.770}\\
		\num{0.085} & \num{0.065} & \num{-0.020} & \num{31} & \num{18.770}\\
		\num{0.159} & \num{0.151} & \num{-0.008} & \num{29} & \num{18.670}\\
		\num{0.150} & \num{0.155} & \num{0.005} & \num{29} & \num{18.670}\\
		\num{0.139} & \num{0.142} & \num{0.003} & \num{29} & \num{18.670}\\
		\num{0.111} & \num{0.120} & \num{0.010} & \num{31} & \num{18.770}\\
		\num{0.177} & \num{0.190} & \num{0.013} & \num{31} & \num{18.770}\\
		\num{0.116} & \num{0.101} & \num{-0.015} & \num{30} & \num{18.720}\\
		\num{0.039} & \num{0.035} & \num{-0.004} & \num{30} & \num{18.720}\\
		\num{0.099} & \num{0.092} & \num{-0.007} & \num{31} & \num{18.770}\\
		\num{0.113} & \num{0.093} & \num{-0.020} & \num{29} & \num{18.670}\\
		\num{0.081} & \num{0.088} & \num{0.006} & \num{29} & \num{18.670}\\
		\num{0.048} & \num{0.057} & \num{0.009} & \num{29} & \num{18.670}\\
		\num{0.071} & \num{0.091} & \num{0.021} & \num{30} & \num{18.720}\\
		\num{0.029} & \num{0.036} & \num{0.007} & \num{29} & \num{18.670}\\
		\num{0.041} & \num{0.030} & \num{-0.012} & \num{30} & \num{18.720}\\
		\hline
	\end{tabular}
	\end{adjustbox}
	\caption{Aus den Messwerten berechnete Steig- und Fallgeschwindigkeiten, deren Differenz, 
	sowie die Temperatur und unkorrigierte Viskosität der Luft \label{tab:Auswertung_Ergebnisse}}
\end{table}


Für die Bestimmung des Radius $r$ der Öltröpfchen nach \cref{eq:} und der korrigierten Werte für 
die Viskosität der Luft $\eta_{L,\text{eff}}$  und der Ladung $q$ der Öltröpfchen nach Cunningham \cref{eq:} und \cref{eq:} 
können nur die Werte verwendet werden, deren Differenzgeschwindigkeit $v_{\text{ab}} - v_{\text{auf}} > 0$ ist.
Ohne diese Einschränkung ergeben sich für den Radius $r$ und die Ladung $q$ komplexe Werte die unphysikalisch und 
nicht zu gebrauchen sind.
Die Werte für den Radius $r$, die korrigierte Viskosität $\eta_{L, \text{eff}}$ und Ladung $q$  befinden sich in 
\cref{tab:Auswertung_Ergebnisse2}.

	 \begin{table}[!h]
	\centering
	\begin{tabular}{|c|c|c|}
		\hline
		Radius & korrigierte Viskosität & Ladung\\
		$r$ [\si{\micro\meter}] & $\eta_{L,\text{eff}}$ [\si{\micro\newton\second\per\square\meter}] & $q$ [\SI{e-19}{\coulomb}]\\
\hline\hline
		\num{0.229} & \num{0.511} & \num{1.611}\\
		\num{0.434} & \num{0.947} & \num{3.737}\\
		\num{0.208} & \num{0.469} & \num{1.947}\\
		\num{0.517} & \num{1.125} & \num{3.861}\\
		\num{0.307} & \num{0.687} & \num{2.838}\\
		\num{0.181} & \num{0.411} & \num{1.284}\\
		\num{0.225} & \num{0.503} & \num{2.192}\\
		\num{0.181} & \num{0.408} & \num{1.632}\\
		\num{0.309} & \num{0.687} & \num{3.383}\\
		\num{0.353} & \num{0.782} & \num{6.144}\\
		\num{0.248} & \num{0.554} & \num{1.981}\\
		\num{0.296} & \num{0.656} & \num{1.476}\\
		\num{0.449} & \num{0.982} & \num{3.442}\\
		\num{0.253} & \num{0.565} & \num{0.770}\\
		\hline
	\end{tabular}
	\caption{Aus den brauchbaren Messwerten berechnete Radien und Ladungen der Tröpfchen,sowie die korrigierte Viskosität der Luft \label{tab:Auswertung_Ergebnisse2}}
\end{table}
    
	 
Die Ergebnisse für die Ladung $q$ sind in \cref{fig:} aufgetragen. Mittels einer linearen Regression,
durchgeführt mit der \emph{Python}-Bibliothek \emph{SciPy} \cite{SciPy},
wurden Horizontalen bestimmt die den Mittelwert nah beieinander liegender Ladungen darstellt.
Außerdem wurden weitere Horizontalen (gestrichelt) in Höhen der ersten vier Vielfachen der
Elementarladung $e_{0} = \SI{1.602e-19}{\coulomb}$ \cite{Mende09} eingezeichnet.


	\includeFigure[scale=0.65]{Grafiken/Messwerte.pdf}{Grafische Darstellung der erhaltenen Ladungen, und gemittelter Ladung von nahe beieinander liegender Ergebnisse}{\label{fig:Auswertung_Ladungen}}


Die so gemittelten Werte für die Vielfachen der Elementarladung und deren Abweichung zu den entsprechenden, 
tatsächlichen Vielfachen der Elementarladung sind in \cref{tab:Auswertung_Ladung} dargestellt.

\begin{table}[!h]
	\centering
	\begin{tabular}{|c|c|c|}
		\hline
		Ladung & Faktor & Abweichung\\
		$n\cdot e_{0}$ [\si{\coulomb}] & $n$ & $1-\frac{e_0}{e_{0,\text{lit}}}$ [\si{\percent}]\\
\hline\hline
		\num{1.612} & \num{1} & \num{0.583}\\
		\num{3.452} & \num{2} & \num{7.740}\\
		\num{6.144} & \num{4} & \num{4.136}\\
		\hline
	\end{tabular}
	\caption{Abweichung der berechneten Vielfachen der Elementarladung vom Literaturwert \label{tab:Auswertung_Ladung}}
\end{table}
    
	