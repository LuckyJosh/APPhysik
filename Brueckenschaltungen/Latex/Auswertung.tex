Im Folgenden sind die aufgenommenen Messwerte und die aus diesen
berechneten Größen vorwiegend tabellarisch aufgetragen. An entsprechender 
Stelle sind Erklärungen zu den Werten und Rechnungen gegeben. 

\subsection{Bestimmung eines Widerstandes mit der Wheatstonebrücke}
\label{sec:Auswertung_Wheatstone}

	Bei dieser Messung wurde der unbekannte Widerstand \emph{Wert 10} vermessen.
	Der am Potentiometer eingestellte Widerstand $R_{3}$, der Quotient aus diesen und den nach \cref{eq:R4} berechneten
	Widerständen $R_{4}$, die jeweiligen
	Abgleichwiderstände $R_{2}$ und die mit Hilfe von \cref{eq:Wheatstone_R} aus diesen
	berechneten Werte für $R_{x}$ sind in \cref{tab:Wheatstone} zu finden.
	
	\begin{table}[!h]
	\centering
	\begin{tabular}{|c|c|c|c|}
		\hline
		Widerstand & Widerstand  & Quotient &  Widerstand\\
		$R_{2}\,[\si{\ohm}]$ & $R_{3}\,[\si{\ohm}]$ & $\frac{R_{3}}{R_{4}}$ & $R_{x}\,[\si{\ohm}]\,\cref{std:Quotient}$\\\hline\hline
		\num{332.0(7)}  & \num{421}  & \num{0.723(4)}  & \num{240(1)} \\
		\num{664(1)}  & \num{266}  & \num{0.361(2)}  & \num{240(1)} \\
		\num{1000(2)}  & \num{195}  & \num{0.241(1)}  & \num{241(1)} \\
		\hline
	\end{tabular}
	\caption{Werte der Messung an der Wheatstonebrücke \label{tab:Wheatstone}}
\end{table}
	
	Der Mittelwert der errechneten Werte für $R_{x}$ ergibt sich aus den Messwerten zu:
	\begin{empheq}{equation*}
		\mean{R_{x}} = \SI{240.4(7)}{\ohm}
	\end{empheq}
	
\subsection{Bestimmung von Kapazitäten mit einer Kapazitätsmessbrücke}
\label{sec:Auswertung_Kapazität}
	In den zwei nachfolgenden Abschnitten werden die Kapazitäten einer 
	idealen und einer realen Kapazität mit Hilfe einer Kapazitätsmessbrücke
	bestimmt.
	
	\subsubsection{Bestimmung einer idealen Kapazität}
	\label{sec:Auswertung_Kapazität_ideal}
		Aus den in \cref{tab:Kapazitaet_ideal} gelisteten Messwerten für die Abgleichkapazitäten
		$C_{2}$, den am Potentiometer eingestellten Widerständen $R_{3}$ und den mit \cref{eq:R4} 
		aus diesen bestimmten $R_{4}$ wurden die ebenfalls in \cref*{tab:Kapazitaet_ideal} dargestellten
		unbekannten Kapazitäten $C_{x}$ (\emph{Wert 3}) unter Verwendung von \cref{eq:Kapazitaet_C} bestimmt. 
		
		\begin{table}[!h]
	\centering
	\begin{tabular}{|c|c|c|c|}
		\hline
		Kapazität & Widerstand & Widerstand & Kapazität\\
		$C_{2}\,[\si{\nano\farad}]$ & $R_{3}\,[\si{\ohm}]$ & $\frac{R_{3}}{R_{4}}$ & $C_{x}\,[\si{\nano\farad}]$\\\hline\hline
		\num{994}  & \num{705}  & \num{2.37(1)}  & \num{420(2)} \\
		\num{750}  & \num{640}  & \num{1.763(9)}  & \num{425(2)} \\
		\num{597}  & \num{589}  & \num{1.423(7)}  & \num{420(2)} \\
		\hline
	\end{tabular}
	\caption{Werte der Messung einer idealen Kapazität an der Kapazitätsmessbrücke \label{tab:Kapazitaet_ideal}}
\end{table}	
	
		Als Mittelwert der unbekannten Kapazität $C_{x}$ erhält man hieraus:
		\begin{empheq}{equation*}
			\mean{C_{x}} = \SI{242(1)}{\nano\farad}
		\end{empheq} 
	
	\subsubsection{Bestimmung einer realen Kapazität}
	\label{sec:Auswertung_Kapazität_real}
		
		Für die Bestimmung einer realen Kapazität (\emph{Wert 9}) wird, anderes als bei der einer idealen Kapazität, ein
		Stellglied $R_{2} = \SI{500(15)}{\ohm}$ benötigt. Die anderen bekannten Größen sind analog zu
		 \cref{sec:Auswertung_Kapazität_ideal} zusammen mit den aus diesen 
		berechneten unbekannten Größen, die Kapazität $C_{x}$ bestimmt durch \cref{eq:Kapazitaet_C} und deren
		 Wirkwiderstand $R_{x}$ bestimmt durch \cref{eq:Kapaziaet_R} in 
		\cref{tab:Kapazitaet_real} eingetragen. 
	
		\begin{table}[!h]
	\centering
	\begin{tabular}{|c|c|c|c|c|}
		\hline
		Kapazität & Widerstand & Quotient & Kapazität & Widerstand\\
		$C_{2}\,[\si{\nano\farad}]$ & $R_{3}\,[\si{\ohm}]$ & $\frac{R_{3}}{R_{4}}\,[\si{}]$ & $C_{x}\,[\si{\nano\farad}]$ & $R_{x}\,[\si{\ohm}]$\\\hline\hline
		\num{994(2)}  & \num{632}  & \num{1.704(9)}  & \num{584(3)}  & \num{852(4)} \\
		\num{750(2)}  & \num{586}  & \num{1.405(7)}  & \num{534(3)}  & \num{703(4)} \\
		\num{597(1)}  & \num{561}  & \num{1.269(6)}  & \num{470(3)}  & \num{635(3)} \\
		\hline
	\end{tabular}
	\caption{Werte der Messung einer idealen Kapazitätan der Kapazitätsmessbrücke \label{tab:Kapazitaet_real}}
\end{table}
		
		Die Mittelwerte der unbekannten Größen $C_{x}$ und $R_{x}$ ergeben sich somit zu:
		\begin{empheq}{equation*}
				\mean{C_{x}} = \SI{529(2)}{\nano\farad} \quad\ \text{und}\ \quad \mean{R_{x}} = \SI{730(2)}{\ohm}
		\end{empheq}  
	
\subsection{Bestimmung von Induktivitäten}
\label{sec:Auswertung_Induktivität}

	Nachfolgend wird eine reale Induktivität (\emph{Wert 16}) zunächst mit Hilfe einer Induktivitätsmessbrücke
	und anschließend mit einer Maxwellbrücke vermessen. Bei beiden Untersuchungen wird 
	ein Stellglied $R_{2} = \SI{1000}{\ohm}$ verwendet.
	
	\subsubsection{Bestimmung Mittels einer Induktivitätsmessbrücke}
	\label{sec:Auswertung_Induktivität_Messbrücke}
	
		Die verwendeten Abgleichinduktivitäten $L_{2}$, der am Potentiometer eingestellten Widerstand
		$R_{3}$ sowie die Quotienten aus diesen und den nach \cref{eq:R4} berechneten Widerständen $R4$
		und die mit Hilfe von \cref{eq:Induktivitaet_L} und \cref{eq:Induktivitaet_R} berechneten unbekannten $L_{x}$
		 und $R_{x}$ sind in \cref{tab:Induktivitaet_Bruecke} zu finden.    
		
		\begin{table}[!h]
	\centering
	\begin{tabular}{|c|c|c|c|c|}
		\hline
		Induktivität & Widerstand & Quotient & Induktivität & Widerstand\\
		$L_{2}\,[\si{\milli\henry}]$ & $R_{3}\,[\si{\ohm}]$ & $\frac{R_{3}}{R_{4}}$ & $L_{x}\,[\si{\milli\henry}]$ & $R_{x}\,[\si{\ohm}]$\\\hline\hline
		\num{20.10(4)}  & \num{305}  & \num{0.437(2)}  & \num{8.78(5)}  & \num{4.4(1)e+02} \\
		\num{27.50(6)}  & \num{321}  & \num{0.471(2)}  & \num{12.94(7)}  & \num{4.7(1)e+02} \\
		\hline
	\end{tabular}
	\caption{Werte der Messung einer realen Induktivität mit einer Induktivitätsmessbrücke \label{tab:Induktivitaets_Bruecke}}
\end{table}
		
		Die aus diesen Werten bestimmten Mittelwerte der Unbekannten Größen sind:
		\begin{empheq}{equation*}
				\mean{L_{x}} = \SI{10.86(4)}{\milli\henry} \quad\ \text{und}\ \quad \mean{R_{x}} = \SI{454(14)}{\ohm}
		\end{empheq}		
		
		
	\subsubsection{Bestimmung Mittels einer Maxwellbrücke}
	\label{sec:Auswertung_Induktivität_Maxwell}
		
		Bei der Bestimmung der Induktivität $L_{x}$ und deren Wirkwiderstand $R_{x}$ werden 
		nur der am Potentiometer eingestellte Widerstand $R_{3}= \SI{210}{\ohm} $, der mit \cref{eq:R4} 
		daraus bestimmte Widerstand $R_{4} = \SI{793}{\ohm}$ und die verwendete Kapazität $C_{4} = 
		\SI{994}{\nano\farad}$  benötigt. Mit \cref{eq:Induktivitaet_Maxwell_L} und \cref{eq:Induktivitaet_Maxwell_R}
		und dem Stellglied $R_{2}$ erhält man:
		\begin{empheq}{equation*}
				{L_{x}} = \SI{0.209(6)}{\henry} \quad\ \text{und}\ \quad {R_{x}} = \SI{265(11)}{\ohm}
		\end{empheq}
		
		
		
\subsection{Bestimmung der Nullfrequenz einer frequenzabhängigen Messbrücke}
\label{sec:Auswertung_Frequenz}

\subsubsection{Bestimmung des Klirrfaktors eins Frequenzgenerators}
\label{sec:Auswertung_Klirrfaktor}