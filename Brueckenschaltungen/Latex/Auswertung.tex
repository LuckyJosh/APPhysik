Im Folgenden sind die aufgenommenen Messwerte und die aus diesen
berechneten Größen vorwiegend tabellarisch aufgetragen. An entsprechender 
Stelle sind Erklärungen zu den Werten und Rechnungen gegeben. 

\subsection{Bestimmung eines Widerstandes mit der Wheatstonebrücke}
\label{sec:Auswertung_Wheatstone}

	Bei dieser Messung wurde der unbekannte Widerstand \emph{Wert 10} vermessen.
	Der am Potentiometer eingestellte Widerstand $R_{3}$, der Quotient aus diesen und den nach \cref{eq:R4} berechneten
	Widerständen $R_{4}$, die jeweiligen
	Abgleichwiderstände $R_{2}$ und die mit Hilfe von \cref{eq:Wheatstone_R} aus diesen
	berechneten Werte für $R_{x}$ sind in \cref{tab:Wheatstone} zu finden.
	
	\begin{table}[!h]
	\centering
	\begin{tabular}{|c|c|c|c|}
		\hline
		Widerstand & Widerstand  & Quotient &  Widerstand\\
		$R_{2}\,[\si{\ohm}]$ & $R_{3}\,[\si{\ohm}]$ & $\frac{R_{3}}{R_{4}}$ & $R_{x}\,[\si{\ohm}]\,\cref{std:Quotient}$\\\hline\hline
		\num{332.0(7)}  & \num{421}  & \num{0.723(4)}  & \num{240(1)} \\
		\num{664(1)}  & \num{266}  & \num{0.361(2)}  & \num{240(1)} \\
		\num{1000(2)}  & \num{195}  & \num{0.241(1)}  & \num{241(1)} \\
		\hline
	\end{tabular}
	\caption{Werte der Messung an der Wheatstonebrücke \label{tab:Wheatstone}}
\end{table}
	
	Der Mittelwert der errechneten Werte für $R_{x}$ ergibt sich aus den Messwerten zu:
	\begin{empheq}{equation}
		\mean{R_{x}} = \SI{240.4(7)}{\ohm}
	\end{empheq}
	
\subsection{Bestimmung von Kapazitäten mit einer Kapazitätsmessbrücke}
\label{sec:Auswertung_Kapazitaet}
	In den zwei nachfolgenden Abschnitten werden die Kapazitäten einer 
	idealen und einer realen Kapazität mit Hilfe einer Kapazitätsmessbrücke
	bestimmt.
	
	\subsubsection{Bestimmung einer idealen Kapazität}
	\label{sec:Auswertung_Kapazität_ideal}
		Aus den in \cref{tab:Kapazitaet_ideal} gelisteten Messwerten für die Abgleichkapazitäten
		$C_{2}$, den am Potentiometer eingestellten Widerständen $R_{3}$ und den mit \cref{eq:R4} 
		aus diesen bestimmten $R_{4}$ wurden die ebenfalls in \cref*{tab:Kapazitaet_ideal} dargestellten
		unbekannten Kapazitäten $C_{x}$ (\emph{Wert 3}) unter Verwendung von \cref{eq:Kapazitaet_C} bestimmt. 
		
		\begin{table}[!h]
	\centering
	\begin{tabular}{|c|c|c|c|}
		\hline
		Kapazität & Widerstand & Widerstand & Kapazität\\
		$C_{2}\,[\si{\nano\farad}]$ & $R_{3}\,[\si{\ohm}]$ & $\frac{R_{3}}{R_{4}}$ & $C_{x}\,[\si{\nano\farad}]$\\\hline\hline
		\num{994}  & \num{705}  & \num{2.37(1)}  & \num{420(2)} \\
		\num{750}  & \num{640}  & \num{1.763(9)}  & \num{425(2)} \\
		\num{597}  & \num{589}  & \num{1.423(7)}  & \num{420(2)} \\
		\hline
	\end{tabular}
	\caption{Werte der Messung einer idealen Kapazität an der Kapazitätsmessbrücke \label{tab:Kapazitaet_ideal}}
\end{table}	
	
		Als Mittelwert der unbekannten Kapazität $C_{x}$ erhält man hieraus:
		\begin{empheq}{equation}
			\mean{C_{x}} = \SI{242(1)}{\nano\farad}
		\end{empheq} 
	
	\subsubsection{Bestimmung einer realen Kapazität}
	\label{sec:Auswertung_Kapazität_real}
		
		Für die Bestimmung einer realen Kapazität (\emph{Wert 9}) wird, anderes als bei der einer idealen Kapazität, ein
		Stellglied $R_{2} = \SI{500(15)}{\ohm}$ benötigt. Die anderen bekannten Größen sind analog zu
		 \cref{sec:Auswertung_Kapazität_ideal} zusammen mit den aus diesen 
		berechneten unbekannten Größen, die Kapazität $C_{x}$ bestimmt durch \cref{eq:Kapazitaet_C} und deren
		 Wirkwiderstand $R_{x}$ bestimmt durch \cref{eq:Kapaziaet_R} in 
		\cref{tab:Kapazitaet_real} eingetragen. 
	
		\begin{table}[!h]
	\centering
	\begin{tabular}{|c|c|c|c|c|}
		\hline
		Kapazität & Widerstand & Quotient & Kapazität & Widerstand\\
		$C_{2}\,[\si{\nano\farad}]$ & $R_{3}\,[\si{\ohm}]$ & $\frac{R_{3}}{R_{4}}\,[\si{}]$ & $C_{x}\,[\si{\nano\farad}]$ & $R_{x}\,[\si{\ohm}]$\\\hline\hline
		\num{994(2)}  & \num{632}  & \num{1.704(9)}  & \num{584(3)}  & \num{852(4)} \\
		\num{750(2)}  & \num{586}  & \num{1.405(7)}  & \num{534(3)}  & \num{703(4)} \\
		\num{597(1)}  & \num{561}  & \num{1.269(6)}  & \num{470(3)}  & \num{635(3)} \\
		\hline
	\end{tabular}
	\caption{Werte der Messung einer idealen Kapazitätan der Kapazitätsmessbrücke \label{tab:Kapazitaet_real}}
\end{table}
		
		Die Mittelwerte der unbekannten Größen $C_{x}$ und $R_{x}$ ergeben sich somit zu:
		\begin{empheq}{equation}
				\mean{C_{x}} = \SI{529(2)}{\nano\farad} \quad\ \text{und}\ \quad \mean{R_{x}} = \SI{730(2)}{\ohm}
		\end{empheq}  
	
\subsection{Bestimmung von Induktivitäten}
\label{sec:Auswertung_Induktivität}

	Nachfolgend wird eine reale Induktivität (\emph{Wert 16}) zunächst mit Hilfe einer Induktivitätsmessbrücke
	und anschließend mit einer Maxwellbrücke vermessen. Bei beiden Untersuchungen wird 
	ein Stellglied $R_{2} = \SI{1000}{\ohm}$ verwendet.
	
	\subsubsection{Bestimmung Mittels einer Induktivitätsmessbrücke}
	\label{sec:Auswertung_Induktivität_Messbrücke}
	
		Die verwendeten Abgleichinduktivitäten $L_{2}$, der am Potentiometer eingestellten Widerstand
		$R_{3}$ sowie die Quotienten aus diesen und den nach \cref{eq:R4} berechneten Widerständen $R4$
		und die mit Hilfe von \cref{eq:Induktivitaet_L} und \cref{eq:Induktivitaet_R} berechneten unbekannten $L_{x}$
		 und $R_{x}$ sind in \cref{tab:Induktivitaet_Bruecke} zu finden.    
		
		\begin{table}[!h]
	\centering
	\begin{tabular}{|c|c|c|c|c|}
		\hline
		Induktivität & Widerstand & Quotient & Induktivität & Widerstand\\
		$L_{2}\,[\si{\milli\henry}]$ & $R_{3}\,[\si{\ohm}]$ & $\frac{R_{3}}{R_{4}}$ & $L_{x}\,[\si{\milli\henry}]$ & $R_{x}\,[\si{\ohm}]$\\\hline\hline
		\num{20.10(4)}  & \num{305}  & \num{0.437(2)}  & \num{8.78(5)}  & \num{4.4(1)e+02} \\
		\num{27.50(6)}  & \num{321}  & \num{0.471(2)}  & \num{12.94(7)}  & \num{4.7(1)e+02} \\
		\hline
	\end{tabular}
	\caption{Werte der Messung einer realen Induktivität mit einer Induktivitätsmessbrücke \label{tab:Induktivitaets_Bruecke}}
\end{table}
		
		Die aus diesen Werten bestimmten Mittelwerte der Unbekannten Größen sind:
		\begin{empheq}{equation}
				\label{eq:LxRx_Brücke}
				\mean{L_{x}} = \SI{10.86(4)}{\milli\henry} \quad\ \text{und}\ \quad \mean{R_{x}} = \SI{454(14)}{\ohm}
		\end{empheq}		
		
		
	\subsubsection{Bestimmung Mittels einer Maxwellbrücke}
	\label{sec:Auswertung_Induktivität_Maxwell}
		
		Bei der Bestimmung der Induktivität $L_{x}$ und deren Wirkwiderstand $R_{x}$ werden 
		nur der am Potentiometer eingestellte Widerstand $R_{3}= \SI{210(6)}{\ohm} $, der mit \cref{eq:R4} 
		daraus bestimmte Widerstand $R_{4} = \SI{793(30)}{\ohm}$ und die verwendete Kapazität $C_{4} = 
		\SI{994}{\nano\farad}$  benötigt. Mit \cref{eq:Induktivitaet_Maxwell_L} und \cref{eq:Induktivitaet_Maxwell_R}
		sowie dem Stellglied $R_{2}$ erhält man:
		\begin{empheq}{equation}
				\label{eq:LxRx_Maxwell}
				{L_{x}} = \SI{0.209(9)}{\henry} \quad\ \text{und}\ \quad {R_{x}} = \SI{265(15)}{\ohm}
		\end{empheq}
		
		
\subsection{Bestimmung der Frequenzabhängigkeit der Brückenspannung}
\label{sec:Auswertung_Frequenz}
	
	Für die Durchführung dieses Versuches wurde eine Wien-Robinson-Brücke, mit folgenden Bauteilen
	verwendet:
	\begin{empheq}{align*}
			R'&= \SI{500}{\ohm}\\
			R &= \SI{332}{\ohm} \\
			C &= \mean{C_{x}} \footnotemark = \SI{242(1)}{\nano\farad}
	\end{empheq}		
\footnotetext{Aus \cref{sec:Auswertung_Kapazität_ideal}}		
	
	Die eingestellten Generatorfrequenzen $\nu$ sind zusammen mit den jeweils gemessenen, doppelten Amplituden der, 
	Brückenspannungen $U_{Br}$ in \cref{tab:Frequenz} zu finden. Die doppelte Amplitude der Generatorspannung wurde zu
	$_{S} = \SI{4.250(1)}{\volt}$ gemessen. 
	
	\begin{table}[!h]
	\centering
	\begin{tabular}{|c|c||c|c|}
		\hline
		Frequenz & Brückenspannung& Frequenz & Brückenspannung\\
		$\nu\,[\si{\hertz}]$ & $U_{Br}\,[\si{\volt}]$ & $\nu\,[\si{\hertz}]$ & $U_{Br}\,[\si{\volt}]$\\\hline\hline
		\num{20.3(1)}  & \num{1.438(1)} & \num{6010.3(1)}  & \num{1.025(1)} \\
		\num{350.4(1)}  & \num{0.944(1)} &\num{7000.3(1)}  & \num{1.050(1)} \\
		\num{670.3(1)}  & \num{0.469(1)} &	\num{8000.3(1)}  & \num{1.075(1)} \\
		\num{800.3(1)}  & \num{0.319(1)} &\num{10030.3(1)}  & \num{1.094(1)} \\
		\num{900.3(1)}  & \num{0.209(1)} &\num{15000.0(1)}  & \num{1.100(1)} \\
		\num{1000.3(1)}  & \num{0.119(1)} &\num{20030.3(1)}  & \num{1.075(1)} \\ 
		\bfseries \num[detect-weight]{1110.3(1)}  & \bfseries \num[detect-weight]{0.033(1)} &	\num{23100.0(1)}  & \num{1.044(1)} \\ 
		\num{1400.3(1)}  & \num{0.178(1)} &\num{25000.0(1)}  & \num{1.038(1)} \\ 
		\num{2000.3(1)}  & \num{0.463(1)} &\num{27000.0(1)}  & \num{1.019(1)} \\
		\num{3200.3(1)}  & \num{0.772(1)} &\num{28000.0(1)}  & \num{1.013(1)} \\
		\num{4000.3(1)}  & \num{0.888(1)} &\num{29000.0(1)}  & \num{1.006(1)} \\
		\num{5000.3(1)}  & \num{0.975(1)} &\num{30100.0(1)}  & \num{0.994(1)} \\
		\hline
	\end{tabular}
	\caption{Generatorfrequenzen und gemessene Brückenspannungen \label{tab:Frequenz}}
\end{table} 
	
	Dabei sind die hervorgehobenen Werte die Frequenz $\nu_{0} = \SI{1110.3(1)}{\hertz}$ bei der die minimale Spannung
	$U_{0} = \SI{0.033(1)}{\volt}$ gemessen wurde. Die durch Division von \cref{eq:Frequenz_0} durch $2\pi$ theoretisch 
	bestimmte Frequenz bei der die 
	Brückenspannung verschwindet ist $\nu_{0,theo} = \SI{1137(3)}{\hertz}$.     
	
	In \cref{fig:WienRobinson} ist der Quotient $\frac{U_{Br}}{U_{S}}$ gegen 
	den Quotienten  $\Omega = \frac{\nu}{\nu_{0}}$ halblogarithmisch aufgetragen, wobei die Zähler dieser Quotienten,
	jeweils die Werte aus \cref{tab:Frequenz} darstellen. Des Weiteren ist noch die Theoriekurve dieses Verlaufes, bestimmt durch radizieren von \cref{eq:UBr_US} eingezeichnet.  
		
	\begin{figure}[!h]
		\centering
		\includegraphics[scale=0.75]{Grafiken/WienRobinson.pdf}
		\caption{Messwerte und Theoriekurve der Spannungs- und Frequenzverhältnisse}
		\label{fig:WienRobinson}
	\end{figure}

\subsubsection{Bestimmung des Klirrfaktors eins Frequenzgenerators}
\label{sec:Auswertung_Klirrfaktor} 
	
	Den Klirrfaktor eines Frequenzgenetrators erhält, unter der Annahme $\displaystyle U_{2}^{2} = \sum_{i=2}^{n} U_{i}^{2}$,
	nach \cref{eq:Klirrfaktor}, durch die Gleichung:
	\begin{empheq}{equation}
		k = \dfrac{U_{2}}{U_{1}}
		\label{eq:Klirrfaktor2}
	\end{empheq}     
	
	Wobei sich die doppelte Amplitude der ersten Oberschwingung $U_{2}$ durch
	\begin{empheq}{equation}
		U_{2} = \dfrac{U_{Br}}{f(2)}\footnotemark
		\label{eq:Oberwelle}
	\end{empheq}
\footnotetext{Wobei $f(\Omega)$ der radizierten Gleichung \cref{eq:UBr_BS} entspricht}
	
	Damit ist die doppelte Amplitude der ersten Oberwelle $U_{2} = \SI{0.221(7)}{\volt}$,
	woraus sich der  Klirrfaktor  $k = \num{6.7(3)}$ ergibt.
	