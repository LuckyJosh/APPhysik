In folgendem Abschnitt sind die während des Versuche aufgenommenen Daten, 
so wie die aus diesen gewonnenen Ergebnisse tabellarisch und mit Hilfe 
von Grafiken dargestellt. An entsprechender Stelle werden Erklärungen zu den 
Messdaten, Rechnungen und Ergebnissen gegeben.

\subsection{Messung der mittleren Reichweite im Abstand 20mm}\label{sec:Messung_I}

	Die Messergebnisse der ersten Messung zu Bestimmung der mittleren Reichweite $R_{m}$
	sind in \cref{tab:Messwerte_I} aufgeführt. Wobei die hervorgehobene Zeile wegen der
	großen Abweichung nicht für die folgende Auswertung genutzt wurde.
	
	
\begin{table}[!h]
	\centering
	\begin{adjustbox}{width=1\textwidth,center}
	\begin{tabular}{|c|c|c|c|c|c||c|}
		\hline
		           Messreihe            &       1        &       2        &       3        &       4        &       5        &      Verschiebung       \\
		             Nr.               &       $$       &       $$       &       $$       &       $$       &       $$       & $D$ [\si{\centi\meter}] \\ \hline
		 \multirow{9}{1.75cm}{Ablenk- \\
		           spannung  \\
		$\quad\! U_{d}\ [\si{\volt}]$} & \num{-20.7(1)} & \num{-33.5(1)} & \num{-35.0(1)} &       -        &       -        &      \num{0.0(1)}       \\
		                               & \num{-17.5(1)} & \num{-27.6(1)} & \num{-28.2(1)} & \num{-35.6(1)} &       -        &      \num{0.6(1)}       \\
		                               & \num{-14.1(1)} & \num{-21.8(1)} & \num{-21.8(1)} & \num{-29.4(1)} & \num{-35.6(1)} &      \num{1.3(1)}       \\
		                               & \num{-10.5(1)} & \num{-16.6(1)} & \num{-15.2(1)} & \num{-21.9(1)} & \num{-26.7(1)} &      \num{1.9(1)}       \\
		                               & \num{-6.9(1)}  & \num{-10.6(1)} & \num{-8.4(1)}  & \num{-14.6(1)} & \num{-17.7(1)} &      \num{2.5(1)}       \\
		                               & \num{-3.4(1)}  & \num{-4.7(1)}  & \num{-2.0(1)}  & \num{-6.8(1)}  & \num{-7.9(1)}  &      \num{3.2(1)}       \\
		                               &  \num{0.4(1)}  &  \num{1.5(1)}  &  \num{5.6(1)}  &  \num{0.8(1)}  &  \num{1.4(1)}  &      \num{3.8(1)}       \\
		                               &  \num{4.7(1)}  &  \num{7.6(1)}  & \num{13.2(1)}  &  \num{8.8(1)}  & \num{11.1(1)}  &      \num{4.4(1)}       \\
		                               &  \num{8.5(1)}  & \num{13.5(1)}  & \num{19.3(1)}  & \num{16.9(1)}  & \num{22.9(1)}  &      \num{5.1(1)}     
		                               \\ \hline
		 \multirow{3}{1.75cm}{Beschl.\\Spannung\\$\quad\!U_{b}$\ [\si{\volt}]}	&&&&&&\\
		 							   & \num{180(5)}   &   \num{300(5)}  & \num{350(5)}  & \num{400(5)} & \num{500(5)} & 
		 							   \\ 
		 							   &&&&&&\\\hline	
	\end{tabular}					   
	\end{adjustbox}
	\caption{Messdaten zur Bestimmung des Zusammenhangs zwischen $U_d$ und $D$ \label{tab:Auswertung_Messdaten_I}}
\end{table}

	
	In \cref{fig:Messdaten_I} sind diese Messwerte grafisch dargestellt, wobei die Gesamtzahl der gemessenen Pulse
	durch Division mit der Messdauer $t = 120 \si{s}$ in die Zerfallsrate umgerechnet wurde. 
	Die effektive Länge, die Strecke die die Alphastrahlung relativ zu Atmosphärendruck,
	zurück gelegt hat berechnet sich nach \cref{eq:}.   
	
	Die mittlere Reichweite $R_{m}$ der Alphastrahlung erhält man nun, indem zunächst eine lineare 
	Regression der Messwerte durchgeführt wird. Die in \cref{fig:Messdaten_I} grau eingezeichneten Messwerte wurden 
	bei dieser Regression nicht verwendet. Mit Hilfe der Python-Bibliothek \emph{SciPy} \cite{SciPy} 
	erhält man aus den Messdaten mit dem Ansatz
	\begin{empheq}{equation}
		A(x) = a \cdot x + b,
	\end{empheq}
	die Regressionsparameter
	\addtocounter{equation}{-1}	
	\begin{subequations}
		\begin{empheq}{align}
			a &= \SI{-20(1)}{\per\second\per\milli\meter} \\
			b &= \SI{280(10)}{\per\second}.
		\end{empheq}
	\end{subequations}
	
	
	Im folgenden Schritt wird eine zur x-Achse parallele Gerade auf halber Höhe des Maximalwerts,
	der gemessenen Zerfallsraten, hier gestrichelt, eingezeichnet.\\
	 
	Die zu bestimmende Reichweite $R_{m}$ lässt sich damit als x-Koordinate des Schnittpunktes dieser beiden Geraden 
	ablesen. Die auf diese Weise ermittelte, mittlere Reichweite beträgt für diese Messdaten 
	$R_{m} = \SI{6.65(1)}{\milli\meter}$.      
	
	\begin{figure}[!h]
		\centering
		\includegraphics[scale=0.7]{Grafiken/MittlereReichweiteI.pdf}
		\caption{Darstellung der Messdaten aus \cref{tab:Messwerte_I} und Bestimmung von $R_{m}$}
		\label{fig:Messdaten_I}
	\end{figure}
	
	Der aus diesen Versuchsdaten zu berechnende Energieverlust $-\od{E}{x}$ wird wegen der besseren
	Messergebnisse im folgenden Unterabschnitt vorgenommen.  
	
	
\subsection{Messung der mittleren Reichweite im Abstand $25$\si{\mm}}\label{sec:Messung_II}


	\begin{table}[!h]
	\centering
	\begin{tabular}{|c|c|c|c|c|}
		\hline
		Druck & Channel Maximum & Energie Maximum & Anzahl Pulse& effektive Länge\\
		$p$ [\si{\milli\bar}] & $Ch_{max}$ & $E_{max}$ [\si{\mega\eV}] & $N$&  $x$ [\si{\milli\meter}] \cref{std:effektiveLaenge}\\
\hline\hline
		\num{0(10)} & \num{559} & \num{4.000} & \num{77188}&\num{0.0(2)} \\
		\num{200(10)} & \num{512} & \num{3.664} & \num{69282}& \num{4.9(3)}\\
		\num{400(10)} & \num{446} & \num{3.191} & \num{58770}& \num{9.9(5)}\\
		\num{450(10)} & \num{468} & \num{3.349} & \num{53517}& \num{11.1(5)}\\
		\num{500(10)} & \num{431} & \num{3.084} & \num{50024}& \num{12.3(6)}\\
		\num{550(10)} & \num{422} & \num{3.020} & \num{46370}& \num{13.6(6)}\\
		\num{600(10)} & \num{419} & \num{2.998} & \num{38034}& \num{14.8(6)}\\
		\num{650(10)} & \num{419} & \num{2.998} & \num{35348}& \num{16.0(7)}\\
		\num{700(10)} & \num{419} & \num{2.998} & \num{26457}& \num{17.3(7)}\\
		\num{750(10)} & \num{419} & \num{2.998} & \num{18744}& \num{18.5(8)}\\
		\num{800(10)} & \num{416} & \num{2.977} & \num{10536}& \num{19.7(8)}\\
		\num{850(10)} & \num{418} & \num{2.991} & \num{5429}& \num{21.0(9)}\\
		\num{900(10)} & \num{412} & \num{2.948} & \num{4797}& \num{22.2(9)}\\
		\num{950(10)} & \num{415} & \num{2.970} & \num{5281}& \num{23(1)}\\
		\num{1000(10)} & \num{421} & \num{3.013} & \num{3660}& \num{25(1)}\\
		\hline
	\end{tabular}
	\caption{Messwerte der Messung im Abstand von $25 \si{mm}$ \label{tab:Messwerte_II}}
\end{table}

	
	
	\begin{figure}[!h]
		\centering
		\includegraphics[scale=0.7]{Grafiken/MittlereReichweiteII.pdf}
		\caption{Darstellung der Messdaten aus \cref{tab:Messwerte_II} und Bestimmung von $R_{m}$}
		\label{fig:Messdaten_II}
	\end{figure}

\subsection{Statistik des radioaktiven Zerfalls}\label{sec:Messung_III}
	
	\begin{figure}[!h]
		\centering
		\includegraphics[scale=0.7]{Grafiken/AktivitaetHistogramm.pdf}
		\caption{Darstellung der Messdaten aus \cref{tab:Messwerte_II} und Bestimmung von $R_{m}$}
		\label{fig:Messdaten_III}
	\end{figure}
		
	\begin{figure}[!h]
		\centering
		\includegraphics[scale=0.7]{Grafiken/VergleichPoisson.pdf}
		\caption{Vergleiche der Messdaten mit der diskreten Poissonverteilung}
		\label{fig:Messdaten_III_Poisson}
	\end{figure}
	
	\begin{figure}[!h]
		\centering
		\includegraphics[scale=0.7]{Grafiken/VergleichGauss.pdf}
		\caption{Vergleich der Messdaten mit der kontinuierlichen Gaussverteilung}
		\label{fig:Messdaten_III_Gauss}
	\end{figure}