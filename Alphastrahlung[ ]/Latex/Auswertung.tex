In folgendem Abschnitt sind die während des Versuche aufgenommenen Daten, 
so wie die aus diesen gewonnenen Ergebnisse tabellarisch und mit Hilfe 
von Grafiken dargestellt. An entsprechender Stelle werden Erklärungen zu den 
Messdaten, Rechnungen und Ergebnissen gegeben.

\subsection{Messung der mittleren Reichweite im Abstand 20mm}\label{sec:Messung_I}

	Die Messergebnisse der ersten Messung zu Bestimmung der mittleren Reichweite $R_{m}$
	sind in \cref{tab:Messwerte_I} aufgeführt. Wobei die hervorgehobene Zeile wegen der
	großen Abweichung nicht für die folgende Auswertung genutzt wurde.
	
	
\begin{table}[!h]
	\centering
	\begin{adjustbox}{width=1\textwidth,center}
	\begin{tabular}{|c|c|c|c|c|c||c|}
		\hline
		           Messreihe            &       1        &       2        &       3        &       4        &       5        &      Verschiebung       \\
		             Nr.               &       $$       &       $$       &       $$       &       $$       &       $$       & $D$ [\si{\centi\meter}] \\ \hline
		 \multirow{9}{1.75cm}{Ablenk- \\
		           spannung  \\
		$\quad\! U_{d}\ [\si{\volt}]$} & \num{-20.7(1)} & \num{-33.5(1)} & \num{-35.0(1)} &       -        &       -        &      \num{0.0(1)}       \\
		                               & \num{-17.5(1)} & \num{-27.6(1)} & \num{-28.2(1)} & \num{-35.6(1)} &       -        &      \num{0.6(1)}       \\
		                               & \num{-14.1(1)} & \num{-21.8(1)} & \num{-21.8(1)} & \num{-29.4(1)} & \num{-35.6(1)} &      \num{1.3(1)}       \\
		                               & \num{-10.5(1)} & \num{-16.6(1)} & \num{-15.2(1)} & \num{-21.9(1)} & \num{-26.7(1)} &      \num{1.9(1)}       \\
		                               & \num{-6.9(1)}  & \num{-10.6(1)} & \num{-8.4(1)}  & \num{-14.6(1)} & \num{-17.7(1)} &      \num{2.5(1)}       \\
		                               & \num{-3.4(1)}  & \num{-4.7(1)}  & \num{-2.0(1)}  & \num{-6.8(1)}  & \num{-7.9(1)}  &      \num{3.2(1)}       \\
		                               &  \num{0.4(1)}  &  \num{1.5(1)}  &  \num{5.6(1)}  &  \num{0.8(1)}  &  \num{1.4(1)}  &      \num{3.8(1)}       \\
		                               &  \num{4.7(1)}  &  \num{7.6(1)}  & \num{13.2(1)}  &  \num{8.8(1)}  & \num{11.1(1)}  &      \num{4.4(1)}       \\
		                               &  \num{8.5(1)}  & \num{13.5(1)}  & \num{19.3(1)}  & \num{16.9(1)}  & \num{22.9(1)}  &      \num{5.1(1)}     
		                               \\ \hline
		 \multirow{3}{1.75cm}{Beschl.\\Spannung\\$\quad\!U_{b}$\ [\si{\volt}]}	&&&&&&\\
		 							   & \num{180(5)}   &   \num{300(5)}  & \num{350(5)}  & \num{400(5)} & \num{500(5)} & 
		 							   \\ 
		 							   &&&&&&\\\hline	
	\end{tabular}					   
	\end{adjustbox}
	\caption{Messdaten zur Bestimmung des Zusammenhangs zwischen $U_d$ und $D$ \label{tab:Auswertung_Messdaten_I}}
\end{table}

	
	In \cref{fig:Messdaten_I} sind diese Messwerte grafisch dargestellt, wobei die Gesamtzahl der gemessenen Pulse
	durch Division mit der Messdauer $\Delta t = 120 \si{s}$ in die Zerfallsrate umgerechnet wurde. 
	Die effektive Länge, die Strecke die die Alphastrahlung relativ zu Atmosphärendruck,
	zurück gelegt hat berechnet sich nach \cref{eq:effektive_laenge}.   
	
	Die mittlere Reichweite $R_{m}$ der Alphastrahlung erhält man nun, indem zunächst eine lineare 
	Regression der Messwerte durchgeführt wird. Die in \cref{fig:Messdaten_I} grau eingezeichneten Messwerte wurden 
	bei dieser Regression nicht verwendet. Mit Hilfe der Python-Bibliothek \emph{SciPy} \cite{SciPy} 
	erhält man aus den Messdaten mit dem Ansatz
	\begin{empheq}{equation}
		A(x) = a \cdot x + b,
	\end{empheq}
	die Regressionsparameter
	\addtocounter{equation}{-1}	
	\begin{subequations}
		\begin{empheq}{align}
			a &= \SI{-20(1)}{\per\second\per\milli\meter} \\
			b &= \SI{280(10)}{\per\second}.
		\end{empheq}
	\end{subequations}
	
	
	Im folgenden Schritt wird eine zur x-Achse parallele Gerade auf halber Höhe des Maximalwerts,
	der gemessenen Zerfallsraten, hier gestrichelt, eingezeichnet.\\
	 
	Die zu bestimmende Reichweite $R_{m}$ lässt sich damit als x-Koordinate des Schnittpunktes dieser beiden Geraden 
	ablesen. Die auf diese Weise ermittelte, mittlere Reichweite beträgt für diese Messdaten 

	\begin{empheq}{equation}
		R_{m} = \SI{6.56(1)}{\milli\meter}.
		\label{eq:Messergebnis_I_R}
	\end{empheq} 
	Durch Umstellen von \cref{eq:reichweite} kann aus diesem Wert 
	die Energie der Alphastrahlung zu 
	\begin{empheq}{equation}
		E_{\alpha} = \SI{1.648(2)}{\mega\eV} 
		\label{eq:Messergebnis_I_E}
	\end{empheq}	
	berechnet werden.\footnote{Der Fehler wurde hierbei durch \eqref{std:Energie} bestimmt.}    

	
	\begin{figure}[!h]
		\centering
		\includegraphics[scale=0.7]{Grafiken/MittlereReichweiteI.pdf}
		\caption{Darstellung der Messdaten aus \cref{tab:Messwerte_I} und Bestimmung von $R_{m}$}
		\label{fig:Messdaten_I}
	\end{figure}
	
	\vspace*{0.5cm}
	Der aus diesen Versuchsdaten zu berechnende Energieverlust $-\od{E}{x}$ wird wegen der besseren
	Messergebnisse im folgenden Unterabschnitt vorgenommen.  
	
	
\subsection{Messung der mittleren Reichweite im Abstand $25$\si{\mm}}\label{sec:Messung_II}
	
	Die bei der Messung im Abstand von \SI{25}{\milli\meter} aufgenommenen Daten sind in \cref{tab:Messwerte_II} 
	dargestellt.

	\begin{table}[!h]
	\centering
	\begin{tabular}{|c|c|c|c|c|}
		\hline
		Druck & Channel Maximum & Energie Maximum & Anzahl Pulse& effektive Länge\\
		$p$ [\si{\milli\bar}] & $Ch_{max}$ & $E_{max}$ [\si{\mega\eV}] & $N$&  $x$ [\si{\milli\meter}] \cref{std:effektiveLaenge}\\
\hline\hline
		\num{0(10)} & \num{559} & \num{4.000} & \num{77188}&\num{0.0(2)} \\
		\num{200(10)} & \num{512} & \num{3.664} & \num{69282}& \num{4.9(3)}\\
		\num{400(10)} & \num{446} & \num{3.191} & \num{58770}& \num{9.9(5)}\\
		\num{450(10)} & \num{468} & \num{3.349} & \num{53517}& \num{11.1(5)}\\
		\num{500(10)} & \num{431} & \num{3.084} & \num{50024}& \num{12.3(6)}\\
		\num{550(10)} & \num{422} & \num{3.020} & \num{46370}& \num{13.6(6)}\\
		\num{600(10)} & \num{419} & \num{2.998} & \num{38034}& \num{14.8(6)}\\
		\num{650(10)} & \num{419} & \num{2.998} & \num{35348}& \num{16.0(7)}\\
		\num{700(10)} & \num{419} & \num{2.998} & \num{26457}& \num{17.3(7)}\\
		\num{750(10)} & \num{419} & \num{2.998} & \num{18744}& \num{18.5(8)}\\
		\num{800(10)} & \num{416} & \num{2.977} & \num{10536}& \num{19.7(8)}\\
		\num{850(10)} & \num{418} & \num{2.991} & \num{5429}& \num{21.0(9)}\\
		\num{900(10)} & \num{412} & \num{2.948} & \num{4797}& \num{22.2(9)}\\
		\num{950(10)} & \num{415} & \num{2.970} & \num{5281}& \num{23(1)}\\
		\num{1000(10)} & \num{421} & \num{3.013} & \num{3660}& \num{25(1)}\\
		\hline
	\end{tabular}
	\caption{Messwerte der Messung im Abstand von $25 \si{mm}$ \label{tab:Messwerte_II}}
\end{table}

		
	Wie in \cref{sec:Messung_I} beschrieben, sind auch die aus diesen Daten berechneten Zerfallsraten
	(Messzeit $\Delta t = \SI{120}{\second}$) in \cref{fig:Messdaten_II} gegen die effektive Länge aufgetragen, um die 
	mittlere Reichweite $R_{m}$ der Alphastrahlung zu bestimmen.\\
	Die dafür notwendige lineare Regression mit dem Ansatz
	\begin{empheq}{equation}
		A(x) = c \cdot x + d, 
	\end{empheq}
	ergibt die Regressionsparameter 
	\addtocounter{equation}{-1}
	\begin{subequations}
		\begin{empheq}{align}
			c &= \SI{-15.2(9)}{\per\second\per\milli\meter}\\
			d &= \SI{271(12)}{\per\second}.
		\end{empheq}
	\end{subequations}

	
	Aus diesen Daten ergibt sich die mittlere Reichweite von 
	\begin{empheq}{equation}
		R_{m} = \SI{8.39(1)}{\milli\meter}.
		\label{eq:Messergebnis_II_R}
	\end{empheq}
	Auch aus diesem Wert lässt sich durch Umstellen der \cref{eq:reichweite} die Energie der 
	Alphastrahlung zu 
	\begin{empheq}{equation}
		E_{\alpha} = \SI{1.943(2)}{\mega\eV} 
		\label{eq:Messergebnis_II_E}
	\end{empheq}
	bestimmen.

	\footnote{Der Fehler wurde hierbei durch \cref{std:Energie} bestimmt.} 
	
	\begin{figure}[!h]
		\centering
		\includegraphics[scale=0.7]{Grafiken/MittlereReichweiteII.pdf}
		\caption{Darstellung der Messdaten aus \cref{tab:Messwerte_II} und Bestimmung von $R_{m}$}
		\label{fig:Messdaten_II}
	\end{figure}


	Durch  das Auftragen der maximal Energien $E_{max}$ aus \cref{tab:Messwerte_II} gegen die effektive Länge $x$,
	in \cref{fig:Messdaten_II_Energie}, ist es möglich mit dem Regressionsansatz 

	\begin{empheq}{equation}
		E(x) = e \cdot x + f, 
	\end{empheq}
	die Funktion des Energieverlaufs mit den Regressionsparametern
	\addtocounter{equation}{-1}
	\begin{subequations}
		\begin{empheq}{align}
			e &= \SI{-0.066(6)}{\mega\eV\per\milli\meter}\\
			f &= \SI{3.97(7)}{\mega\eV}
		\end{empheq}
	\end{subequations}	
	zu bestimmen.\\
	Durch Differentiation der auf diese Weise bestimmten Funktion $E(x)$ erhält man den gesuchten Energieverlust 
	\begin{empheq}{equation}
		-\od{E}{x} = \num{0.066(6)} \si{\mega\eV\per\milli\meter}.
	\end{empheq}	

	\begin{figure}[!h]
		\centering
		\includegraphics[scale=0.7]{Grafiken/EnergieVerlauf.pdf}
		\caption{Darstellung der maximal Energie in Abhängigkeit der effektiven Länge}
		\label{fig:Messdaten_II_Energie}
	\end{figure}
	

\subsection{Statistik des radioaktiven Zerfalls}\label{sec:Messung_III}
	
	In \cref{tab:Messwerte_III} sind die  während der Messzeit 
	$\Delta t = \SI{10}{\second}$ aufgenommenen Zerfallsraten aufgelistet.
	\begin{table}[!h]
	\centering
	\begin{tabular}{|c|c||c|c||c|c|}
		\hline
		Messung & Zerfallsrate & Messung & Zerfallsrate & Messung & Zerfallsrate\\
		 Nr. & $A$ [\si{\per\second}]& Nr. & $A$ [\si{\per\second}] & Nr. & $A$ [\si{\per\second}] \\
\hline\hline
		\num{1} & \num{569}& \num{35} & \num{1223}&	\num{69} & \num{658}\\ 
		\num{2} & \num{616}& \num{36} & \num{1308}&	\num{70} & \num{638}\\
		\num{3} & \num{590}& \num{37} & \num{841}&	\num{71} & \num{649}\\				
		\num{4} & \num{566}& \num{38} & \num{647}&	\num{72} & \num{688}\\			
		\num{5} & \num{599}& \num{39} & \num{598}&	\num{73} & \num{707}\\				 
		\num{6} & \num{627}& \num{40} & \num{633}&	\num{74} & \num{896}\\				
		\num{7} & \num{612}& \num{41} & \num{645}&	\num{75} & \num{844}\\					 
		\num{8} & \num{604}& \num{42} & \num{593}&	\num{76} & \num{593}\\						 
		\num{9} & \num{596}& \num{43} & \num{573}&	\num{77} & \num{746}\\					 
		\num{10} & \num{618}& \num{44} & \num{603}& \num{78} & \num{1089}\\
		\num{11} & \num{617}& \num{45} & \num{561}& \num{79} & \num{674}\\
		\num{12} & \num{621}& \num{46} & \num{645}&	\num{80} & \num{754}\\
		\num{13} & \num{634}& \num{47} & \num{611}& \num{81} & \num{750}\\
		\num{14} & \num{603}& \num{48} & \num{575}&	\num{82} & \num{834}\\
		\num{15} & \num{588}& \num{49} & \num{590}& \num{83} & \num{795}\\
		\num{16} & \num{669}& \num{50} & \num{591}& \num{84} & \num{703}\\
		\num{17} & \num{633}& \num{51} & \num{692}& \num{85} & \num{858}\\
		\num{18} & \num{630}& \num{52} & \num{601}&	\num{86} & \num{702}\\ 
		\num{19} & \num{618}& \num{53} & \num{655}&	\num{87} & \num{636}\\ 
		\num{20} & \num{590}& \num{54} & \num{776}& \num{88} & \num{599}\\
		\num{21} & \num{593}& \num{55} & \num{783}&	\num{89} & \num{1206}\\
		\num{22} & \num{595}& \num{56} & \num{997}&	\num{90} & \num{664}\\ 
		\num{23} & \num{575}& \num{57} & \num{838}&	\num{91} & \num{609}\\
		\num{24} & \num{600}& \num{58} & \num{683}&	\num{92} & \num{660}\\
		\num{25} & \num{682}& \num{59} & \num{681}&	\num{93} & \num{662}\\
		\num{26} & \num{636}& \num{60} & \num{815}& \num{94} & \num{663}\\
		\num{27} & \num{639}& \num{61} & \num{628}& \num{95} & \num{689}\\
		\num{28} & \num{670}& \num{62} & \num{670}& \num{96} & \num{644}\\
		\num{29} & \num{601}& \num{63} & \num{652}& \num{97} & \num{648}\\
		\num{30} & \num{628}& \num{64} & \num{623}& \num{98} & \num{604}\\
		\num{31} & \num{652}& \num{65} & \num{620}& \num{99} & \num{613}\\
		\num{32} & \num{599}& \num{66} & \num{719}& \num{100}& \num{581}\\
		\num{33} & \num{663}& \num{67} & \num{858}&	  & \\
		\num{34} & \num{617}& \num{68} & \num{597}& & \\
		
		\hline
	\end{tabular}
	\caption{Anzahl der gemessenen Impulse \label{tab:Messwerte_III}}
\end{table}

	Der Mittelwert und die Standardabweichung dieser Messwerte berechnen sich zu
	\begin{empheq}{align}
		\mean{A} &= \SI{681(14)}{\per\second}\\
		\sigma_{A} &= \SI{136}{\per\second}\\
	\end{empheq}
	
	Diese Messwert sind in \cref{fig:Messdaten_III} in einem Histogramm aufgetragen,
	in dem die Balkenbreite $\Delta N = \SI{20}{\per\second}$ gewählt wurde.
	
	In den Abbildungen \ref{fig:Messdaten_III_Poisson} und \ref{fig:Messdaten_III_Gauss}
	ist das Histogramm aus \cref{fig:Messdaten_III} noch einmal im Vergleich zu einer 
	diskreten Poisson- bzw. kontinuierlichen Gauß-Verteilung mit dem Mittelwert und der 
	Standardabweichung der Messdaten dargestellt.
	
	
	\begin{figure}[!h]
		\centering
		\includegraphics[scale=0.7]{Grafiken/AktivitaetHistogramm.pdf}
		\caption{Darstellung der Messdaten aus \cref{tab:Messwerte_II} und Bestimmung von $R_{m}$}
		\label{fig:Messdaten_III}
	\end{figure}
		
	\begin{figure}[!h]
		\centering
		\includegraphics[scale=0.7]{Grafiken/VergleichPoisson.pdf}
		\caption{Vergleiche der Messdaten mit der diskreten Poissonverteilung}
		\label{fig:Messdaten_III_Poisson}
	\end{figure}
	
	\begin{figure}[!h]
		\centering
		\includegraphics[scale=0.7]{Grafiken/VergleichGauss.pdf}
		\caption{Vergleich der Messdaten mit der kontinuierlichen Gaussverteilung}
		\label{fig:Messdaten_III_Gauss}
	\end{figure}
