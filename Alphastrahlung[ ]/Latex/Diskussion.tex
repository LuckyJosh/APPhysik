Im Folgenden werden die erhaltenen Versuchsergebnisse noch einmal
abschließend diskutiert und dabei auf ihre Plausibilität hin untersucht.\\

Die in den ersten beiden Messreihen erhaltenen Ergebnisse für die 
Energie der Alphastrahlung (\cref{eq:Messergebnis_I_E} und \cref{eq:Messergebnis_II_E})
weisen zwar eine Abweichung von $\Delta E_{\alpha} = \SI{1.07}{\mega\eV}$ auf. 
Ein möglicher Grund hierfür könnte sein, dass der Threshhold des Vielkanalanalysator
für die zweite Messung verringert wurde, wodurch eine größere Anzahl an Impulsen gemessen
wurde.   
Hinzu kommt jedoch, dass das Ergebnis der zweiten Messung nicht unterhalb der
in der Anleitung \cite{V701} gegebenen Grenze $E_{\alpha} = \SI{2.5}{\mega\eV}$ liegt.
Ein Vergleich der Messdaten zeigt jedoch, dass die im Abstand $x_{0} = \SI{25}{\milli\meter}$ 
gemessenen Daten wesentlich weniger Abweichungen enthalten als die des kürzeren Abstandes,
wodurch die in \cref{sec:Messung_II}  berechnete Energie \cref{eq:Messergebnis_II_E} 
der Alphastrahlung plausibler zu seien scheint. \\

Der Vergleich zwischen dem in \cref{fig:Messdaten_III} dargestellten Histogramm und 
den Poissonverteilung in \cref{fig:Messdaten_III_Poisson} und der Gaußverteilung in
\cref{fig:Messdaten_III_Gauss} zeigen, dass die Statistik des radioaktiven Zerfalls
keiner der beiden Verteilungen entspricht.  \\
Im Vergleich zur Poissonverteilung zeigen die Messwerte einige Unterschiede auf,
wie zum Beispiel die unsymmetrische Verteilung und die daraus resultierende,
vom Mittelwert abweichende Stelle des Peaks. Jedoch ist der Verlauf der Messdaten für
Zerfallsraten $A > A_{max}$ dem der Poissonverteilung  für $A > \mean{A}$ sehr ähnlich, 
sodass diese zur Näherungsweisen Beschreibung des Verlaufs verwendet werden kann. 
Auch die Gaußverteilung, zeigt die gleichen Unterschiede und Gemeinsamkeiten zu den Messwerten
auf wie schon die Poissonverteilung. Lediglich der langsamere Abfall gegen $\pm \inf$ 
ist für die Beschreibung des radioaktiven Zerfalls eher ungeeignet.\\
Dadurch ist zu vermuten, dass die Poissonverteilung im Vergleich zu Gaußverteilung den Radioaktiven Zerfall besser beschreibt. 