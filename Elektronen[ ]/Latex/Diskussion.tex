In diesem Abschnitt werden die in der Auswertung erhaltenen Ergebnisse 
noch einmal aufgegriffen und durch Vergleich mit Literaturwerten 
oder ähnlichem auf ihre Plausibilität hin untersucht. Dabei 
wird, falls erforderlich auch Bezug auf die Versuchsdurchführung 
und den Aufbau genommen. \\

Der aus den fünf Messreihen \cref{tab:Auswertung_Messdaten_I} erhaltenen Proportionalitätsfaktor\\  $ \alpha = \SI{30.3(8)}{\centi\meter}$ \cref{eq:Auswertung_Ergebnis_I} 
stellt mit einer relativen Abweichung von $\Delta_{rel}(\alpha) = \SI{7(3)}{\percent}$ zum 
Theoriewert \cref{eq:Auswertung_Ergebnis_I_theo} ein, für die relativ geringe Anzahl an Messungen, 
plausibles und genaues Ergebnis dar.
Ein Grund für diese Abweichung könnten die für die Herleitung verwendeten, vereinfachten 
annahmen sein, wie die als parallel angenommenen Platten des Ablenkkondensators.
Durch die vorgenommene Mittelung wurde zwar ein Abstand $d$ errechnet, der als Abstand 
eines parallelen Plattenpaares angenommen werden kann, doch stellt auch dieser  
nur ein Näherungswert dar, wodurch es zu Abweichungen zum Versuchsergebnis kommt.


Die Ermittlung der Sinusfrequenz $f_{sin}$ zeigt in allen vier Fällen 
\cref{tab:Auswertung_Oszilloskop} eine Übereinstimmung bis zur ersten und
in drei Fällen sogar bis zu zweiten Nachkommastelle. Wodurch die
Plausibilität der Relation \cref{eq:Theorie_freqVerhältnis} gezeigt ist.

Die Ergebnisse der Messung der spezifischen Ladung des Elektrons 
\cref{tab:Auswertung_Parameter_B} zeigen, in bis auf einem Fall, 
im Vergleich zum Literaturwert
$e_{spez,lit} = \SI{1.7588e11}{\coulomb\per\kilo\g}$\cite{Mende09} nur geringe Abweichungen.
Dies ist auch daran zu erkennen, dass der Literaturwert in den berechneten 
Fehlern der Ergebnisse liegt.
Ein möglicher Grund für die Abweichung des letzten Messwertes könnte sein, 
dass die eingestellte Beschleunigungsspannung nicht der Tatsächlichen entsprach,
diese Möglichkeit wird durch die geringe Abweichung der Messwerte in der 3. und 4.
Messreihe in \cref{tab:Auswertung_Messdaten_II} bekräftigt.

Das bestimmte Erdmagnetfeld mit $B_{total} =  \SI{52(2)}{\micro\tesla}$ 
\cref{eq:Auswertung_Btotal} stellt sich im Vergleich mit dem Literaturwert von
$B_{total,lit} = \SI{47}{\micro\tesla}$ \cite{GGU} für Mitteleuropa,
in Anbetracht der Tatsache, dass nur eine Messung durchgeführt wurde, als 
plausibles und gutes Ergebnis heraus. \\

Alles in allem lässt sich somit sagen, dass alle Ergebnisse dieses Versuchs
plausibel und mehr noch, in Anbetracht der geringen Zahl an Wiederholungen,
relativ genau sind. Dies lässt darauf schließen das der verwendete Versuchsaufbau
zur Messung der erhaltenen Größen gut geeignet und dabei außerdem noch unanfällig
für äußere Störungen ist.   
