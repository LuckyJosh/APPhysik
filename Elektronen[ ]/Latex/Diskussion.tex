In diesem Abschnitt werden die in der Auswertung erhaltenen Ergebnisse 
noch einmal aufgegriffen und durch Vergleich mit Literaturwerten 
oder ähnlichem auf ihre Plausibilität hin untersucht. Dabei 
wird, falls erforderlich auch Bezug auf die Versuchsdurchführung 
und den Aufbau genommen. \\

Der aus den fünf Messreihen \cref{tab:Auswertung_Messdaten_I} erhaltenen Proportionalitätsfaktor\\  $ \alpha = \SI{30.3(8)}{\centi\meter}$ \cref{eq:Auswertung_Ergebnis_I} 
stellt mit einer relativen Abweichung von $\Delta_{rel}(\alpha) = \SI{7(3)}{\percent}$ zum 
Theoriewert \cref{eq:Auswertung_Ergebnis_I_theo} ein, für die relativ geringe Anzahl an Messungen, 
plausibles und genaues Ergebnis dar.\\
% Wodurch die Abweichungen?

Die Ermittlung der Sinusfrequenz $f_{sin}$ zeigt in allen vier Fällen 
\cref{tab:Auswertung_Oszilloskop} eine Übereinstimmung bis zur ersten und
in drei Fällen sogar bis zu zweiten Nachkommastelle. Wodurch die
Plausibilität der Relation \cref{eq:Theorie_} gezeigt ist.\\

Die Ergebnisse der Messung der spezifischen Ladung des Elektrons 
\cref{tab:Auswertung_Parameter_B} zeigen, in bis auf einem Fall, 
im Vergleich zum Literaturwert
$e_{spez,lit} = \SI{1.7588e11}{\coulomb\per\kilo\g}$\cite{Mende09} nur geringe Abweichungen.
Dies ist auch daran zu erkennen, dass der Literaturwert in den berechneten 
Fehlern der Ergebnisse liegt.\\   
% Grund für letzte Abweichung ?

Das bestimmte Erdmagnetfeld mit $B_{total} =  \SI{52(2)}{\micro\tesla}$ 
\cref{eq:Auswertung_Btotal} stellt sich im Vergleich mit dem Literaturwert von
$B_{total,lit} = \SI{47}{\micro\tesla}$ \cite{GGU} für Mitteleuropa,
in Anbetracht der Tatsache, dass nur eine Messung durchgeführt wurde, als 
plausibles und gutes Ergebnis heraus. 

