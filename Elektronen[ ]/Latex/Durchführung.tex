\section{Durchführung}
\subsection{E-Feld}

Für ein festes $U_B$ werden die Ablenkspannungen gemessen, die die Verschiebung des Leuchtpunkts genau auf eine der Gitternetzlinien (Abstand $\frac{1}{4}$ inch.) auf dem Schirm bewirken. Dieser Vorgang wird für
sechs verschiedene Beschleunigungsspannungen wiederholt.
Jetzt werden zwei Funktionengeneratoren (Sinus- und Sägezahnspannung) auf die X- und Y-Eingänge gelegt und Frequenzen gemessen, bei denen sich das Verhältnis 

\begin{equation}
n \cdot \nu_\text{Sägezahn} = m \cdot \nu_\text{Sinus} \quad n, m\in\mathbb{N}
\end{equation}

einstellt.

\subsection{B-Feld}

Ein homogenes Magnetfeld wird mit Hilfe eines Helmholtzspulenpaares erzeugt. Zunächst muss
mit einem Inklinatorium die Horizontalkomponente des Erdmagnetfelds bestimmt und die
Messapparatur in diese Richtung gedreht werden, damit das Erdmagnetfeld den Versuch nicht verfälscht,
danach wird mit konstanter Beschleunigungsspannung die Strahlverschiebung in Abhängigkeit vom Magnetfeld gemessen, das bei einem Helmholtz-Spulenpaar vom durchfließenden Strom abhängt.
\begin{equation}
B = \mu_0\frac{8 N I}{\sqrt{125}R}
\end{equation}

Dabei ist N die Windungszahl (hier: 20) und R den Spulenradius (hier: 0,28 m).
Des weiteren wird die Stärke des Erdmagnetfeldes durch Ausrichten des Leuchtpunkts auf die Mitte des Schirmes und dann durch drehen des Versuchsaufbaus um 90° bestimmt. Dabei verschiebt sich der Elektronenstrahl. Nun wird das von der Helmholtz-Spule erzeugte Magnetfeld
gerade so hoch geregelt, dass der Elektronenpunkt in der Mitte des Schirmes ist. Aus diesem Wert des Magnetfeldes und dem Inklinationswinkel des Erdmagnetfeldes, der ebenfalls mit dem Inklinatorium bestimmt wird, kann man die Stärke des Erdmagnetfeldes errechnen.
