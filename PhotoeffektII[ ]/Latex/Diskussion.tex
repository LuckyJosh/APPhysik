Im folgenden Abschnitt werden die in der Auswertung erhaltenen Ergebnisse
noch einmal abschließend diskutiert und dabei auf ihre Plausibilität hin
überprüft. Dabei wird auch Bezug auf den Versuchsaufbau und die -durchführung genommen.\\

Der erhaltene Wert für den Quotienten aus planckschem Wirkungsquantum und 
Elementarladung \cref{val:Auswertung_h_e0} weißt mit einer relativen Abweichung
von \SI{180}{\percent} einen sehr großen Fehler 
zum Literaturwert \SI{4,136e-15}{\volt\second} \cite{SciPy} auf.
Dieser Fehler lässt sich durch die Genauigkeit der durchgeführten Messungen und dens
somit abweichenden Dispersionsspannungen begründen. 
Dies fällt wegen der geringen Größenordnung der Naturkonstante $ \sfrac{h}{e_{0}} $ ins
Gewicht, da so geringe Abweichungen der Messwerte für Strom und Spannung die 
in der Größenordnung \num{e01} und \num{e-03} aufgenommen wurden, große 
Auswirkung auf kleinere Größenordnungen haben.


  