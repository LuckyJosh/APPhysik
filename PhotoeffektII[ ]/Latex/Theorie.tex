Eine Leuchtdiode Funktioniert wie die Umkehrung des Photoeffektes. Eine Leuchtdiode besteht aus Halbleiterverbindungen,
 durch die der Strom in Licht umgewandelt wird.
Eine Diode besteht aus zwei verschiedenen Halbleitern, aus einem p-Leiter und einem n-Leiter. Vom n-Leiter bewegen sich
die Elektronen zum p-Leiter und treffen dort auf positive Ladungen, sogenannte Löcher. An der Grenzschicht gleichen 
sich die positiven und die negativen Ladungen aus. Durch diese Elektronendiffusion wird, in der nähe der Grenzschicht 
der n-Leiter positiv und der p-Leiter negativ geladen, wodurch sich ein elektrisches Feld in np-Richtung (von n-Leiter 
nach p-Leiter) einstellt. Dieses Feld verhindert ab einer bestimmten Stärke die weitere Diffusion von Elektronen.

Legt man nun eine elektrische Spannung und damit ein elektrisches Feld in pn-Richtung an, so wirkt dieses Feld 
dem E-Feld in der Grenzschicht entgegen und schwächt dieses ab. Hat das externe Feld eine bestimmte Grenzspannung,
die Diffusionsspannung $U_{D}$, überschritten ist ein Ladungsaustausch zwischen n- und p-Leiter wieder möglich,
es kann somit ein Strom durch die Diode fließen.  
Wenn an die Diode eine Spannung angelegt wird werden die Elektronen mit der Energie
\begin{align}
E=e_0U
\end{align}
beschleunigt. Die Elektronen geben dabei ihre Energie in Form von Strahlung ab oder und regen die Atome im 
Gitterverbund des Halbleiters durch Abgabe von Energie zu schwingen an.
Die Energiegleichung hat damit die Form
\begin{align}
e_0U=hf+A_S,
\end{align}
wobei $A_S$ die Energie ist, die an die Gitteratome abgegeben wird.
Die Wellenlänge $\lambda$ beziehungsweise die Frequenz $f$ des emittierten Photonen, hängt im wesentlichen von
der Energiedifferenz zwischen Leitungs- und Valenzband des verwendeten Halbleiters ab.\\ 



